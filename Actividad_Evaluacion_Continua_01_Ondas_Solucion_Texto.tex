\documentclass[14pt]{extarticle}
\usepackage[utf8]{inputenc}
\usepackage[T1]{fontenc}
\usepackage[spanish,es-lcroman]{babel}
\usepackage{amsmath}
\usepackage{amsthm}
\usepackage{physics}
\usepackage{tikz}
\usepackage{float}
\usepackage[autostyle,spanish=mexican]{csquotes}
\usepackage[per-mode=symbol]{siunitx}
\usepackage{gensymb}
\usepackage{multicol}
\usepackage{enumitem}
\usepackage[left=2.00cm, right=2.00cm, top=2.00cm, 
     bottom=2.00cm]{geometry}
\usepackage{Estilos/ColoresLatex}
\usepackage{makecell}

\newcommand{\textocolor}[2]{\textbf{\textcolor{#1}{#2}}}
\DeclareSIUnit[number-unit-product = {\,}]\cal{cal}

%\renewcommand{\questionlabel}{\thequestion)}
\decimalpoint
\sisetup{bracket-numbers = false}

\title{\vspace*{-2cm} Actividad 1 - Propiedades de las ondas \\ \Large{Solución} \vspace{-5ex}}
\date{}

\begin{document}
\maketitle

\section*{Ejercicios a resolver.}

\begin{enumerate}
\item Calcular la magnitud de la velocidad con la que se propaga una onda cuya frecuencia es de \SI{150}{\hertz} y su longitud de onda es de \SI{7}{\meter}.

\textbf{Solución:}

\begin{minipage}[t]{0.3\linewidth}
\noindent
\textocolor{red}{Datos:}
\begin{align*}
f &= \SI{150}{\hertz} \\
\lambda &= \SI{7}{\meter} \\
v &= \, ?
\end{align*}
\end{minipage}
\hspace{1cm}
\begin{minipage}[t]{0.3\linewidth}
\noindent
\textocolor{red}{Expresión}:
\begin{align*}
v = \lambda \, f
\end{align*}
\end{minipage}

\begin{minipage}{0.5\linewidth}
\noindent
\textocolor{red}{Sustitución}:
\begin{align*}
v = (\SI{7}{\meter})(\SI{150}{\hertz}) = \SI{1050}{\meter\per\second}
\end{align*}
\end{minipage}
\item Una lancha sube y baja por el paso de las olas cada \SI{4}{\second}, entre cresta y cresta hay una distancia de \SI{15}{\meter}. ¿Cuál es la magnitud de la velocidad con que se mueven las olas?

\textbf{Solución:}

\begin{minipage}[t]{0.3\linewidth}
\noindent
\textocolor{red}{Datos:}
\begin{align*}
T &= \SI{4}{\second} \\
\lambda &= \SI{15}{\meter} \\
v &= \, ?
\end{align*}
\end{minipage}
\hspace{1cm}
\begin{minipage}[t]{0.3\linewidth}
\noindent
\textocolor{red}{Expresión(es)}:
\begin{align*}
f &= \dfrac{1}{T} \\
v &= \lambda \, f
\end{align*}
\end{minipage}

\begin{minipage}{0.5\linewidth}
\noindent
\textocolor{red}{Sustitución}:
\begin{align*}
f &= \dfrac{1}{\SI{4}{\second}} = \SI{0.25}{\hertz} \\
v &= (\SI{15}{\meter})(\SI{0.25}{\hertz}) = \SI{3.75}{\meter\per\second}
\end{align*}
\end{minipage}
\item La cresta de una onda producida en la superficie libre de un líquido avanza \SI{0.5}{\meter\per\second}. Si tiene una longitud de onda de \SI{4d-1}{\meter}, calcula su frecuencia.

\textbf{Solución:}

\begin{minipage}[t]{0.35\linewidth}
\noindent
\textocolor{red}{Datos:}
\begin{align*}
v &= \SI{0.5}{\meter\per\second} \\[0.5em]
\lambda &= \SI{4d-1}{\meter} \\[0.5em]
f &= \, ?
\end{align*}
\end{minipage}
\hspace{1cm}
\begin{minipage}[t]{0.35\linewidth}
\noindent
\textocolor{red}{Expresión(es)}:
\begin{align*}
v &= \lambda \, f \\[0.5em]
f &= \dfrac{v}{\lambda}
\end{align*}
\end{minipage}

\begin{minipage}{0.5\linewidth}
\noindent
\textocolor{red}{Sustitución}:
\begin{align*}
f &= \dfrac{\SI{0.5}{\meter\per\second}}{\SI{4d-1}{\meter}} = \SI{1.25}{\hertz}
\end{align*}
\end{minipage}
\item Por una cuerda tensa se propagan ondas con una frecuencia de \SI{30}{\hertz} y una rapidez de propagación de \SI{10}{\meter\per\second}. ¿Cuál es su longitud de onda?

\textbf{Solución:}

\begin{minipage}[t]{0.35\linewidth}
\noindent
\textocolor{red}{Datos:}
\begin{align*}
f &= \SI{30}{\hertz} \\[0.5em]
v &= \SI{10}{\meter\per\second} \\[0.5em]
\lambda &= \, ?
\end{align*}
\end{minipage}
\hspace{1cm}
\begin{minipage}[t]{0.35\linewidth}
\noindent
\textocolor{red}{Expresión(es)}:
\begin{align*}
v &= \lambda \, f \\[0.5em]
\lambda &= \dfrac{v}{f}
\end{align*}
\end{minipage}

\begin{minipage}{0.5\linewidth}
\noindent
\textocolor{red}{Sustitución}:
\begin{align*}
\lambda = \dfrac{\SI{10}{\meter\per\second}}{\SI{30}{\hertz}} = \SI{0.33}{\meter}
\end{align*}
\end{minipage}
\item Calcula la frecuencia y el periodo de las ondas producidas en una cuerda de guitarra, si tienen una rapidez de propagación de \SI{12}{\meter\per\second} y su longitud de onda es de \SI{0.06}{\meter}.

\textbf{Solución:}

\begin{minipage}[t]{0.35\linewidth}
\noindent
\textocolor{red}{Datos:}
\begin{align*}
v &= \SI{12}{\meter\per\second} \\[0.5em]
\lambda &= \SI{0.06}{\meter} \\[0.5em]
T &= \, ?
\end{align*}
\end{minipage}
\hspace{1cm}
\begin{minipage}[t]{0.35\linewidth}
\noindent
\textocolor{red}{Expresión(es)}:
\begin{align*}
v &= \lambda \, f \hspace{0.3cm} \Rightarrow \hspace{0.3cm} f = \dfrac{v}{\lambda} \\[0.5em]
T &= \dfrac{1}{f}    
\end{align*}
\end{minipage}

\begin{minipage}{0.5\linewidth}
\noindent
\textocolor{red}{Sustitución}:
\begin{align*}
f &= \dfrac{\SI{12}{\meter\per\second}}{\SI{0.06}{\meter}} = \SI{200}{\hertz} \\[0.5em]
T &= \dfrac{1}{\SI{200}{\hertz}} = \SI{0.005}{\second}
\end{align*}
\end{minipage}
\item Determina la frecuencia de las ondas que se transmiten por una cuerda tensa, cuya rapidez de propagación es de \SI{200}{\meter\per\second} y su longitud de onda es de \SI{0.7}{\meter}.

\textbf{Solución:}

\begin{minipage}[t]{0.35\linewidth}
\noindent
\textocolor{red}{Datos:}
\begin{align*}
v &= \SI{200}{\meter\per\second} \\[0.5em]
\lambda &= \SI{0.7}{\meter} \\[0.5em]
f &= \, ?
\end{align*}
\end{minipage}
\hspace{1cm}
\begin{minipage}[t]{0.35\linewidth}
\noindent
\textocolor{red}{Expresión(es)}:
\begin{align*}
v &= \lambda \, f \hspace{0.3cm} \Rightarrow \hspace{0.3cm} f = \dfrac{v}{\lambda}
\end{align*}
\end{minipage}

\begin{minipage}{0.5\linewidth}
\noindent
\textocolor{red}{Sustitución}:
\begin{align*}
f &= \dfrac{\SI{200}{\meter\per\second}}{\SI{0.7}{\meter}} = \SI{285.71}{\hertz}
\end{align*}
\end{minipage}

\item ¿Cuál es la rapidez con que se propaga una onda longitudinal en un resorte, cuando su frecuencia es de \SI{180}{\hertz} y su longitud de onda es de \SI{0.8}{\meter}?

\textbf{Solución:}

\begin{minipage}[t]{0.35\linewidth}
\noindent
\textocolor{red}{Datos:}
\begin{align*}
f&= \SI{180}{\hertz} \\[0.5em]
\lambda &= \SI{0.06}{\meter} \\[0.5em]
v &= \, ?
\end{align*}
\end{minipage}
\hspace{1cm}
\begin{minipage}[t]{0.35\linewidth}
\noindent
\textocolor{red}{Expresión}:
\begin{align*}
v &= \lambda \, f    
\end{align*}
\end{minipage}

\begin{minipage}{0.5\linewidth}
\noindent
\textocolor{red}{Sustitución}:
\begin{align*}
v &= (\SI{0.03}{\meter})(\SI{90}{\hertz}) = \SI{2.7}{\meter\per\second}
\end{align*}
\end{minipage}

\item Se produce un tren de ondas en una cuba de ondas, entre cresta y cresta hay una distancia de \SI{0.03}{\meter}, con una frecuencia de \SI{90}{\hertz}. ¿Cuál es la magnitud de la velocidad de propagación de las ondas?

\textbf{Solución:}

\begin{minipage}[t]{0.35\linewidth}
\noindent
\textocolor{red}{Datos:}
\begin{align*}
\lambda &= \SI{0.03}{\meter} \\[0.5em]
f &= \SI{90}{\hertz} \\[0.5em]
v &= \, ?
\end{align*}
\end{minipage}
\hspace{1cm}
\begin{minipage}[t]{0.35\linewidth}
\noindent
\textocolor{red}{Expresión}:
\begin{align*}
v &= \lambda \, f   
\end{align*}
\end{minipage}

\begin{minipage}{0.5\linewidth}
\noindent
\textocolor{red}{Sustitución}:
\begin{align*}
v &= (\SI{0.03}{\meter})(\SI{90}{\hertz}) = \SI{2.7}{\meter\per\second}
\end{align*}
\end{minipage}
\end{enumerate}
\end{document}