\documentclass[12pt, letter]{exam}
\usepackage[utf8]{inputenc}
\usepackage[T1]{fontenc}
\usepackage[spanish]{babel}
\usepackage[autostyle,spanish=mexican]{csquotes}
\usepackage{amsmath}
\usepackage{amsthm}
\usepackage{physics}
\usepackage{tikz}
\usepackage{float}
\usepackage{siunitx}
\usepackage{multicol}
\usepackage{paralist}
\usepackage{enumitem}
\usepackage[left=2.00cm, right=2.00cm, top=2.00cm, 
     bottom=2.00cm]{geometry}
\usepackage{pdfpages}
\usepackage{circuitikz}

% \renewcommand{\questionlabel}{\thequestion}
\decimalpoint

\setlength{\belowdisplayskip}{-0.5pt}

\usepackage{tasks}
\settasks{
    label=\Alph*), 
    label-align=left,
    item-indent={20pt}, 
    column-sep={4pt},
    label-width={16pt},
}

\sisetup{per-mode=symbol}
\footer{}{\thepage}{}

\begin{document}
% \includepdf[pages={1}]{Caratula_Examen_Parcial_03_PU_Fisica_4_01.pdf}

% \newpage
\begin{center}
\textbf{Cada problema vale 1 punto, incluidos los ejercicios de ejecución.}
\end{center}

\begin{questions}

    \section{Ondas.}
    A continuación se te presenta la definición de un concepto, la palabra faltante está en una de las opciones de respuesta, relaciona cada definición con la palabra.

    \question 

\end{questions}
%     \section{(6 puntos) Hidrodinámica.}

%     \question \textbf{(1 punto)} En la hidrodinámica un fluido se considera ideal cuando:
%     \begin{parts}
%         \part La densidad es constante.
%         \part Las fuerzas son no conservativas.
%         \part Las partículas solo tienes movimiento de traslación.
%         \part La temperatura del fluido varía con la velocidad.
%     \end{parts}
%     \question \textbf{(1 punto)} \label{Problema_01} \textbf{Ejercicio de ejecución: } En el cuerpo humano el flujo sanguíneo es de $5$ litros de sangre por minuto. ¿Cuál es el área de la sección transversal de la aorta, si la sangre en ese vaso tiene una velocidad de \SI{28}{\centi\meter\per\second}?
%     \begin{tasks}(4)
%         \task \SI{3.10}{\square\centi\meter}
%         \task \SI{2.77}{\square\centi\meter}
%         \task \SI{1.77}{\square\centi\meter}
%         \task \SI{2.97}{\square\centi\meter}
%     \end{tasks}
%     \question \textbf{(1 punto)} Bernoulli al estudiar la relación entre la presión y la velocidad de un líquido que circula por una tubería, encontró que la presión es \rule{2cm}{0.1mm} si su velocidad es \rule{2cm}{0.1mm}:
%     \begin{tasks}(4)
%         \task alta - cero.
%         \task baja - máxima.
%         \task baja - alta.
%         \task alta - alta.
%     \end{tasks}
%     \question \textbf{(1 punto)} La ecuación de Bernoulli que se muestra a continuación contiene tres términos de cada lado de la igualdad:
%     \begin{align*}
%     \underbrace{P_{1}}_{\text{\large{a}}} + \underbrace{\rho \, g \, h_{1}}_{\text{\large{b}}} + \underbrace{\dfrac{1}{2} \rho \, v_{1}^{2}}_{\text{\large{c}}} = P_{2} + \rho \, g \, h_{2} + \dfrac{1}{2} \rho \, v_{2}^{2}
%     \end{align*}
        
%     \vspace{0.5em}
%     \begin{inparaenum}[I)]
%             \item Energía cinética. \quad \quad
%             \item Energía potencial. \quad \quad
%             \item Potencia. \quad \quad
%             \item Presión.
%     \end{inparaenum}

%     Selecciona la respuesta que relaciona los tres términos con las cantidades físicas correspondientes.
%     \begin{tasks}
%         \task a - III, b - I, c - II
%         \task a - IV, b - I, c - II
%         \task a - IV, b - II, c - I
%         \task a - III, b - II, c - I
%     \end{tasks}
%     \question \textbf{(1 punto)} En la siguiente figura se presenta el flujo \rule{2cm}{0.1mm} en un conducto.
%     \begin{figure}[H]
%         \centering
%         \includegraphics[scale=0.8]{Flujo_02_Laminar.png}
%     \end{figure}
%     \begin{tasks}(4)
%         \task laminar
%         \task turbulento
%         \task seccionado
%         \task estático
%     \end{tasks}
%     \question \textbf{(1 punto)} El número de Reynolds es una cantidad adimensional que nos indica cuándo un flujo \rule{2cm}{0.1mm} pasa a ser un flujo \rule{2cm}{0.1mm}.
%     \begin{tasks}
%         \task laminar - seccionado
%         \task estático - turbulento
%         \task laminar  - turbulento
%         \task turbulento - laminar
%     \end{tasks}
    
%     \section{(9 puntos) Electricidad.}

%     \question \textbf{(1 punto)} El flujo de las partículas cargadas en un conductor es lo que se conoce como \rule{2cm}{0.1mm}.
%     \begin{tasks}(4)
%         \task Resistencia.
%         \task Voltaje.
%         \task Potencia.
%         \task Corriente.
%     \end{tasks}
%     \question \textbf{(1 punto)} Relaciona las unidades de las siguientes cantidades, una columna con la otra.
%     \\
%     \begin{minipage}[t]{0.4\linewidth}
%         \begin{enumerate}[label=\arabic*)]
%             \item Voltaje.
%             \item Resistencia.
%             \item Corriente.
%         \end{enumerate}
%     \end{minipage}
%     \begin{minipage}[t]{0.4\linewidth}
%         \begin{enumerate}[label=\alph*)]
%             \item Ohm (\si{\ohm})
%             \item Watt (\si{\watt})
%             \item Ampere (\si{\ampere})
%             \item Volt (\si{\volt})
%         \end{enumerate}
%     \end{minipage}
%     \begin{tasks}(4)
%         \task 1-d, 2-b, 3-a
%         \task 1-a, 2-d, 3-c
%         \task 1-d, 2-a, 3-c
%         \task 1-a, 2-c, 3-b
%     \end{tasks}
%     \question \textbf{(1 punto)} \label{Problema_02} \textbf{Ejercicio de ejecución: } Determina la intensidad de la corriente eléctrica a través de una resistencia de \SI{150}{\kilo\ohm} al aplicarle una diferencia de potencial de \SI{220}{\volt}.
%     \begin{tasks}(4)
%         \task \SI{1.466}{\milli\ampere}
%         \task \SI{1.966}{\ampere}
%         \task \SI{2.066}{\milli\ampere}
%         \task \SI{2.066}{\ampere}
%     \end{tasks}
%     \question \textbf{(1 punto)} En el siguiente circuito eléctrico se tienen $n$ resistencias del mismo valor conectadas en serie, ¿cuál es el valor de la resistencia total $(R_{T})$ del circuito entre los puntos $A^{\prime}$ y $B^{\prime}$?
%     \begin{center}
%     \begin{circuitikz}[american voltages]
%         \draw 
%             (0, 0) node [anchor=east] {$A^{\prime}$}
%             to[short, o-] (1, 0)
%             % (0, 0) to[V=10V] (0, 4)
%             to [R, l=\mbox{$R$}] (3, 0)
%             to [R, l=\mbox{$R$}] (5, 0)
%             to [R, l=\mbox{$R$}] (7, 0)
%             -- (8.5, 0) node[midway,fill=white,inner sep=5,scale=1.2] {$.\,.\,.$} (10.5, 0)
%             ;
%             \draw (8.5, 0) to [R, l=\mbox{$R$}] (10.5, 0)
%                 to[short, -o] (12, 0)
%                 node [anchor=west] {$B^{\prime}$};
%     \end{circuitikz}  
%     \end{center}
%     \begin{tasks}(4)
%         \task $R_{T} = n^{2} \, R$
%         \task $R_{T} = n + R$
%         \task $R_{T} = n \, R$
%         \task $R_{T} = \dfrac{R}{n}$
%     \end{tasks}
%     \question \textbf{(1 punto)} En un circuito serie $RC$, se tiene que $R = \SI{5}{\kilo\ohm}$ y $C = \SI{3}{\micro\farad}$. ¿Cuál es la constante de tiempo del circuito?
%     \begin{tasks}(4)
%         \task \SI{15}{\second}
%         \task \SI{1.5d-3}{\second}
%         \task \SI{0.015}{\second}
%         \task \SI{0.35}{\second}
%     \end{tasks}
%     \question \textbf{(1 punto)} En un circuito \rule{2cm}{0.1mm} la corriente se adelanta \ang{90} con respecto al voltaje:
%     \begin{tasks}(4)
%         \task $R \, C$
%         \task $R \, L$
%         \task $R \, L \, C$
%         \task $R$
%     \end{tasks}
%     \question \textbf{(1 punto)} \label{Problema_03} \textbf{Ejercicio de ejecución: } Calcula la reactancia inductiva del siguiente inductor:
%     \begin{center}
%         \begin{circuitikz}[american voltages]
%             \draw 
%                 (0, 0) to[short, o-] (1, 0)
%                 to [L, l=\mbox{$L=\SI{1.2}{\henry}$}] (3, 0)
%                 to[short, -o] (4, 0);
%             \node at (2, -0.75) {$\omega = \SI{377}{\radian\per\second}$};
%         \end{circuitikz}  
%     \end{center}
%     \begin{tasks}(4)
%         \task \SI{500.9}{\ohm}
%         \task \SI{452.4}{\ohm}
%         \task \SI{370.7}{\ohm}
%         \task \SI{246.1}{\ohm}
%     \end{tasks}
%     \question \textbf{(1 punto)} En un circuito en serie una resistencia $R = \SI{8}{\ohm}$ y un inductor $L = \SI{0.02}{\henry}$ están conectados a una fuente de voltaje $v = \num 283 \, \sin 300 \, t$ Volts. ¿Cuál es el valor de la impedancia compleja?
%     \begin{tasks}(4)
%         \task $Z {=} \SI{16}{\ohm} + j \, \SI{16}{\ohm}$
%         \task $Z {=} \SI{16}{\ohm} - j \, \SI{16}{\ohm}$
%         \task $Z {=} \SI{8}{\ohm} - j \, \SI{6}{\ohm}$
%         \task $Z {=} \SI{8}{\ohm} + j \, \SI{6}{\ohm}$
%     \end{tasks}
%     \question \textbf{(1 punto)} Una resistencia $R = \SI{27.5}{\ohm}$ y un capacitor $C = \SI{66.7}{\micro\farad}$ se conectan en serie. La diferencia de potencial es $V = \num{50} \, \cos 1500 \, t$ \, Volts. ¿Cuál es la magnitud de la impedancia $Z$?
%     \begin{tasks}(4)
%         \task \SI{14.99}{\ohm}
%         \task \SI{980.95}{\ohm}
%         \task \SI{32.32}{\ohm}
%         \task \SI{224.70}{\ohm}
%     \end{tasks}

% \end{questions}

% \vspace*{1cm}
% \textbf{\huge{Formulario.}}
% \begin{table}[H]
%     \centering
%     \setlength{\tabcolsep}{40pt}
%     \renewcommand{\arraystretch}{2.5}
%     \begin{tabular}{c  c}
%         \multicolumn{2}{c}{Hidrodinámica} \\
%         $G = \dfrac{V}{t}$ & $G = A \, v$ \\
%         $\text{Flujo} = \dfrac{m}{t}$ & $\text{Flujo} = \dfrac{\rho \, V}{t}$ \\
%         $\text{Flujo} = G \, \rho$ & \\
%         $v_{1} \, A_{1} = v_{2} \, A_{2}$ & $v = \sqrt{2 \, g \, h}$ \\
%         \multicolumn{2}{c}{$P_{1} + \rho \, g \, h_{1} + \dfrac{1}{2} \rho \, v_{1}^{2} = P_{2} + \rho \, g \, h_{2} + \dfrac{1}{2} \rho \, v_{2}^{2}$} \\ \hline
%         \multicolumn{2}{c}{Electricidad} \\
%         $V = I \, R$ & $R_{T} = R_{1} + R_{2} + R_{3} + \ldots$ \\
%         $\dfrac{1}{R_{T}} = \dfrac{1}{R_{1}} + \dfrac{1}{R_{2}} + \dfrac{1}{R_{3}} + \ldots$ & $\tau = R \, C$ \\
%         $X_{C} = \dfrac{1}{\omega \, C}$ & $X_{L} = \omega\, L$ \\
%         $Z = R + j \, X_{L}$ & $Z = R - j \, X_{C}$ \\
%         $\abs{Z} = \sqrt{R^{2} + (X_{L})^{2}}$ & $\abs{Z} = \sqrt{R^{2} + (X_{C})^{2}}$ \\
% \end{tabular}
% \end{table}

% \newpage

% En este espacio deberás de incluir el desarrollo completo de los Problemas de Ejecución. El problema se califica de la siguiente manera: \textbf{a) Datos: 0.25 puntos}, \textbf{b) Expresión(es): 0.25 puntos}, \textbf{c) Sustitución: 0.25 puntos} y \textbf{d) Manejo de unidades: 0.25 puntos}.

% \vspace*{0.5cm}
% Solución al Problema de Ejecución \ref{Problema_01}:

% \vspace*{4cm}
% \rule{0.9\textwidth}{0.3mm}

% Solución al Problema de Ejecución \ref{Problema_02}:

% \vspace*{4.5cm}
% \rule{0.9\textwidth}{0.3mm}

% Solución al Problema de Ejecución \ref{Problema_03}:


\end{document}