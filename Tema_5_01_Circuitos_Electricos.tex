\documentclass[14pt]{beamer}
\usepackage{./Estilos/BeamerUVM}
\usepackage{./Estilos/ColoresLatex}
\usepackage{circuitikz}
\input{Preambulos/preambulo_Beamer_Cambridge_wolverine}
% \usefonttheme{serif}
\usepackage[clock]{ifsym}
\DeclareSIUnit\erg{erg}
\DeclareSIUnit[number-unit-product = {\,}]\cal{cal}

\sisetup{per-mode=symbol}
\resetcounteronoverlays{saveenumi}

% Macro para agregar el logo de UVM en cada slide de la presentación

\addtobeamertemplate{frametitle}{}{%
\begin{tikzpicture}[remember picture,overlay]
\coordinate (logo) at ([xshift=-1.5cm,yshift=-0.8cm]current page.north east);
% \fill[devryblue] (logo) circle (.9cm);
% \clip (logo) circle (.75cm);
\node at (logo) {\includegraphics[width=2.1cm]{Imagenes/logo_UVM.png}};
\end{tikzpicture}}


\title{\Large{Circuitos eléctricos} \\ \normalsize{Física IV (Área II)}}
\date{}

\begin{document}
\maketitle

\section*{Contenido}
\frame[allowframebreaks]{\frametitle{Contenido} \tableofcontents[currentsection, hideallsubsections]}

\section{Circuitos eléctricos}
\frame{\tableofcontents[currentsection, hideothersubsections]}
\subsection{Definición}

\begin{frame}
\frametitle{¿Qué es un circuito eléctrico?}
Los circuitos eléctricos son sistemas que permiten la circulación de corriente eléctrica para realizar diversas funciones, \pause como proporcionar energía, realizar tareas específicas o transmitir información.
\end{frame}
\begin{frame}
\frametitle{¿Qué es un circuito eléctrico?}
Un circuito eléctrico típico consta de varios componentes interconectados, \pause y la teoría de circuitos eléctricos se basa en principios fundamentales que describen cómo interactúan estos componentes.
\end{frame}

\subsection{Componentes}

\begin{frame}
\frametitle{Fuente de Alimentación}
Proporciona la energía eléctrica al circuito.
\\
\bigskip
\pause
Puede ser una batería, un generador o cualquier dispositivo que suministre voltaje.
\end{frame}
\begin{frame}
\frametitle{Cables o Conductores}
Permiten que la corriente eléctrica fluya desde la fuente de alimentación hasta los dispositivos y componentes del circuito.
\end{frame}
\begin{frame}
\frametitle{Cables o Conductores}    
Los cables están hechos de materiales conductores como el cobre.
\end{frame}
\begin{frame}
\frametitle{Interruptor}
Abre o cierra el circuito, permitiendo o deteniendo el flujo de corriente.
\end{frame}
\begin{frame}
\frametitle{Interruptor}
Es un dispositivo de control que puede estar en posición abierta (apagado) o cerrada (encendido).
\end{frame}
\begin{frame}
\frametitle{Resistencia}
Limita el flujo de corriente en el circuito.
\\
\bigskip
\pause
Se mide en ohms $(\Omega)$ y se representa con el símbolo $R$.
\end{frame}
\begin{frame}
\frametitle{Capacitor}
Almacena carga eléctrica y libera esa carga cuando sea necesario.
\\
\bigskip
\pause
Su unidad es el Farad $(F)$ y se representa con el símbolo $C$.
\end{frame}
\begin{frame}
\frametitle{Bobina o Inductor}
Almacena energía en forma de campo magnético cuando la corriente fluye a través de ella.
\end{frame}
\begin{frame}
\frametitle{Bobina o Inductor}
Se mide en Henry $(H)$ y se representa con el símbolo $L$.
\end{frame}
\begin{frame}
\frametitle{Diodo}
Permite el flujo de corriente en una dirección \pause y bloquea en la dirección opuesta.
\\
\bigskip
\pause
Se utiliza para rectificación, entre otros propósitos.
\end{frame}

\subsection{Tipos de circuitos}

\begin{frame}
\frametitle{Circuitos en serie}
Los componentes están conectados en una única trayectoria, \pause la corriente tiene que pasar a través de cada componente.
\end{frame}
\begin{frame}
\frametitle{Circuito en serie}
\begin{figure}
\centering
\begin{circuitikz}
 \draw (0,0) to[R=$R_{1}$] (2,0) to[R=$R_{2}$] (4,0) to[R=$R_{3}$] (6,0);
 \draw (0,0)--(0,-2) to[battery1={9 V}] (6,-2)--(6,0);
\end{circuitikz}
\end{figure}
\end{frame}
\begin{frame}
\frametitle{La corriente en un circuito serie}
Como se mencionó, la corriente en el circuito serie es la misma en todos los elementos que lo componen.
\pause
\begin{align*}
I_{T} = I_{1} = I_{2} = I_{3} = \ldots
\end{align*}
\end{frame}
\begin{frame}
\frametitle{El voltaje en un circuito serie}
El voltaje total se \enquote{divide} en cada componente del circuito, es decir:
\pause
\begin{align*}
V_{T} = V_{1} + V_{2} + V_{3} + \ldots 
\end{align*}
\end{frame}
\begin{frame}
\frametitle{La resistencia total}
En un circuito en serie, la resistencia total es:
\pause
\begin{align*}
R_{T} = R_{1} + R_{2} + R_{3} + \ldots
\end{align*}
\end{frame}
\begin{frame}
\frametitle{Circuitos en paralelo}
Los componentes están conectados en múltiples trayectorias, \pause y la corriente se divide entre ellos.
\end{frame}
\begin{frame}
\frametitle{Circuito en paralelo}
\vspace*{-1cm}
\begin{figure}
\centering
\begin{circuitikz}[scale=0.9]
    \draw (0,0) to[R=$R_{1}$] (2,0);
    \draw (0,-2) to[R=$R_{2}$] (2,-2);
    \draw (0,-4)  to[R=$R_{3}$] (2,-4);
    \draw (0,0)--(0,-6) (2,0)--(2,-6);
    \draw (0,-6) to[battery1={9 V}] (2,-6);
   \end{circuitikz}
\end{figure}
\end{frame}
\begin{frame}
\frametitle{La resistencia total}
En el circuito en paralelo, la resistencia total es:
\pause
\begin{align*}
\dfrac{1}{R_{T}} = \dfrac{1}{R_{1}} + \dfrac{1}{R_{2}} + \dfrac{1}{R_{3}} + \ldots
\end{align*}
\end{frame}
\begin{frame}
\frametitle{El voltaje en un circuito en paralelo}
En un circuito en paralelo, el voltaje es el mismo en cada componente:
\pause
\begin{align*}
V_{T} = V_{1} = V_{2} = V_{3} = \ldots
\end{align*}
\end{frame}
\begin{frame}
\frametitle{La corriente en un circuito paralelo}
La corriente en el circuito paralelo se \enquote{divide} en todos los elementos que lo componen.
\end{frame}
\begin{frame}
\frametitle{El corriente en un circuito paralelo}
La corriente total se \enquote{divide} en cada componente del circuito, es decir:
\pause
\begin{align*}
I_{T} = I_{1} + I_{2} + I_{3} + \ldots 
\end{align*}
\end{frame}

\subsection{Circuito mixto}

\begin{frame}
\frametitle{Circuitos mixtos}
Contienen combinaciones de circuitos en serie y en paralelo.
\end{frame}
\begin{frame}
\frametitle{Circuito mixto}
\vspace*{-1cm}
\begin{figure}
\centering
\begin{circuitikz}
\draw (0,0) to[battery1={9 V}]    (0,4) % La fuente de voltaje
            to[R=$R_{1}$] (4,4)
            to[R=$R_{2}$]  (4,0) % La resistencia
            to[short]       (3,0)
            to[short]       (3,.5)
            to [R, a=$R_{3}$] (1,.5)  |-  (0,0) % <---
        (3,0) to[short]       (3,-.5)
            to [R=$R_{4}$] (1,-.5)             % <----
            to [short]      (1,0);
    \end{circuitikz}
\end{figure}
\end{frame}
\begin{frame}
\frametitle{¿Cómo se resuelve?}
Para obtener la resistencia total:
\setbeamercolor{item projected}{bg=lava,fg=white}
\setbeamertemplate{enumerate items}{%
\usebeamercolor[bg]{item projected}%
\raisebox{1.5pt}{\colorbox{bg}{\color{fg}\footnotesize\insertenumlabel}}%
}
\begin{enumerate}[<+->]
\item Se resuelven primero aquellas partes del circuito que estén conectadas en paralelo.
\item Se anota como una resistencia equivalente, con un subíndice: $R_{E1}$.
\seti
\end{enumerate}
\end{frame}
\begin{frame}
\frametitle{Circuito mixto}
\vspace*{-1cm}
\begin{figure}
\centering
\begin{circuitikz}
\draw (0,0) to[battery1={9 V}]    (0,4) % La fuente de voltaje
            to[R=$R_{1}$] (4,4)
            to[R=$R_{2}$]  (4,0) % La resistencia
            to[short]       (3,0)
            % to[short]       (3,.5)
            to [R, a=$R_{E1}$] (1, 0)  |-  (0,0) % <---
            to [short]      (1, 0);
    \end{circuitikz}
\end{figure}
\end{frame}
\begin{frame}
\frametitle{¿Cómo se resuelve?}
\setbeamercolor{item projected}{bg=lava,fg=white}
\setbeamertemplate{enumerate items}{%
\usebeamercolor[bg]{item projected}%
\raisebox{1.5pt}{\colorbox{bg}{\color{fg}\footnotesize\insertenumlabel}}%
}
\begin{enumerate}[<+->]
\conti
\item Buscando dejar un circuito ya sea en serie, o en paralelo, representando una sola resistencia: la resistencia total $R_{T}$
\seti
\end{enumerate}
\end{frame}
\begin{frame}
\frametitle{Circuito mixto}
\vspace*{-1cm}
\begin{figure}
\centering
\begin{circuitikz}
\draw (0, 0) to[battery1={9 V}]    (0, 3) % La fuente de voltaje
            % to[R=$R_{1}$] (4,4)
            to[short]      (3, 3)
            to[R=$R_{T}$]  (3, 0) % La resistencia
            % to[short]       (3,.5)
            % to [R, a=$R_{E1}$] (1, 0)  |-  (0,0) % <---
            to [short]      (0, 0);
    \end{circuitikz}
\end{figure}
\end{frame}
\begin{frame}
\frametitle{¿Cómo se resuelve?}
\setbeamercolor{item projected}{bg=lava,fg=white}
\setbeamertemplate{enumerate items}{%
\usebeamercolor[bg]{item projected}%
\raisebox{1.5pt}{\colorbox{bg}{\color{fg}\footnotesize\insertenumlabel}}%
}
\begin{enumerate}[<+->]
\conti
\item Una vez conocida la resistencia total $(R_{T})$, por la ley de Ohm, se conocerá la corriente total $(I_{T})$
\seti
\end{enumerate}
\end{frame}
\begin{frame}
\frametitle{¿Cómo se resuelve?}
\setbeamercolor{item projected}{bg=lava,fg=white}
\setbeamertemplate{enumerate items}{%
\usebeamercolor[bg]{item projected}%
\raisebox{1.5pt}{\colorbox{bg}{\color{fg}\footnotesize\insertenumlabel}}%
}
\begin{enumerate}[<+->]
\conti
\item Se procede a calcular la corriente y voltaje de cada componente de manera inversa a como se resolvió el circuito.
\end{enumerate}
\end{frame}
\begin{frame}
\frametitle{Resolviendo un ejercicio}
Veamos el procedimiento para calcular el voltaje y la corriente que circula por cada una de las resistencias del siguiente circuito mixto.
\\
\bigskip
\pause
Todas las resistencias valen \SI{10}{\ohm}
\end{frame}
\begin{frame}[plain]
\begin{figure}
\centering
\begin{circuitikz}
    \draw (0, 0) to[battery1={10 V}] (0, 4) % La fuente de voltaje
        to[R=$R_{1}$] (4, 4)
        to[short] (4, 3.5)
        to[short] (3.5, 3.5) |- (4.5, 3.5)
        to[short] (4.5, 3)
        to[R=$R_{3}$] (4.5, 1)
        to[R=$R_{4}$] (4.5, -1)
        to[short] (4, -1)
        to[short] (4, -2)
        to[short] (3, -2)
        to[short] (3, -2.5) |- (3, -1.5)
        to[R, l_=$R_{5}$] (1, -1.5)
        to[short] (1, -2) -- (0, -2) -- (0, 0);
    
    \draw (1, -2) to[short] (1, -2.5)
        to[R, l_=$R_{6}$] (3, -2.5);
    
    \draw (3.5, 3.5) to[R, l_=$R_{2}$] (3.5, -1) -- (4, -1);
\end{circuitikz}
\end{figure}    
\end{frame}
\begin{frame}
\frametitle{Comenzando a simplificar}
Nos fijamos en las resistencias $R_{3}$ y $R_{4}$ que están conectadas en serie.
\\
\bigskip
\pause
La suma de esas resistencias será la resistencia equivalente $R_{E1}$.
\end{frame}
\begin{frame}[plain]
\begin{figure}
\centering
\begin{circuitikz}
    \draw (0, 0) to[battery1={10 V}] (0, 4) % La fuente de voltaje
        to[R=$R_{1}$] (4, 4)
        to[short] (4, 3.5)
        to[short] (3.5, 3.5) |- (4.5, 3.5)
        to[short] (4.5, 3)
        to[R=$R_{3}$] (4.5, 1)
        to[R=$R_{4}$] (4.5, -1)
        to[short] (4, -1)
        to[short] (4, -2)
        to[short] (3, -2)
        to[short] (3, -2.5) |- (3, -1.5)
        to[R, l_=$R_{5}$] (1, -1.5)
        to[short] (1, -2) -- (0, -2) -- (0, 0);
    
    \draw (1, -2) to[short] (1, -2.5)
        to[R, l_=$R_{6}$] (3, -2.5);
    
    \draw (3.5, 3.5) to[R, l_=$R_{2}$] (3.5, -1) -- (4, -1);

    \draw [red, thick] (4.2, 3.25) rectangle (4.8, -1.25);
\end{circuitikz}
\end{figure}    
\end{frame}
\begin{frame}
\frametitle{La resistencia equivalente}
Al estar $R_{3}$ y $R_{4}$ en serie, la resistencia equivalente es:
\pause
\begin{align*}
R_{E1} = R_{1} + R_{2} = \SI{10}{\ohm} + \SI{10}{\ohm} = \SI{20}{\ohm}
\end{align*}
\end{frame}
\begin{frame}
\frametitle{Segunda resistencia equivalente}
Ahora tenemos que $R_{2}$ y $R_{E1}$ están en paralelo, por lo que procedemos a calcular la resistencia equivalente $R_{E2}$
\end{frame}
\begin{frame}[plain]
\begin{figure}
\centering
\begin{circuitikz}
    \draw (0, 0) to[battery1={10 V}] (0, 4) % La fuente de voltaje
        to[R=$R_{1}$] (4, 4)
        to[short] (4, 3.5)
        to[short] (3.5, 3.5) |- (4.5, 3.5)
        to[short] (4.5, 3)
        to[R=$R_{E1}$] (4.5, -1)
        % to[R=$R_{4}$] (4.5, -1)
        to[short] (4, -1)
        to[short] (4, -2)
        to[short] (3, -2)
        to[short] (3, -2.5) |- (3, -1.5)
        to[R, l_=$R_{5}$] (1, -1.5)
        to[short] (1, -2) -- (0, -2) -- (0, 0);
    
    \draw (1, -2) to[short] (1, -2.5)
        to[R, l_=$R_{6}$] (3, -2.5);
    
    \draw (3.5, 3.5) to[R, l_=$R_{2}$] (3.5, -1) -- (4, -1);

    \draw [red, thick] (3.2, 3.75) rectangle (4.8, -1.25);
\end{circuitikz}
\end{figure}    
\end{frame}
\begin{frame}
\frametitle{La resistencia equivalente $R_{E2}$}
\vspace*{-1cm}
Como $R_{2}$ y $R_{E1}$ están en paralelo, la resistencia equivalente $R_{E2}$ es:
\pause
\begin{eqnarray*}
\begin{aligned}
\dfrac{1}{R_{E2}} &= \dfrac{1}{R_{2}} + \dfrac{1}{R_{E1}} \\[0.1em] \pause
R_{E2} &= \dfrac{1}{\dfrac{1}{R_{2}} + \dfrac{1}{R_{E1}}} = \pause \dfrac{1}{\dfrac{1}{\SI{10}{\ohm}} + \dfrac{1}{\SI{20}{\ohm}}} = \pause \dfrac{1}{\dfrac{3}{20} \si{\ohm}} = \\[1em] \pause
R_{E2} &= \SI{6.66}{\ohm}
\end{aligned}
\end{eqnarray*}
\end{frame}
\begin{frame}[plain]
    \begin{figure}
    \centering
    \begin{circuitikz}
    \draw (0, 0) to[battery1={10 V}] (0, 4) % La fuente de voltaje
        to[R=$R_{1}$] (4, 4)
        % to[short] (4, 3.5)
        % to[short] (3.5, 3.5) |- (4.5, 3.5)
        to[short] (4, 3)
        to[R=$R_{E2}$] (4, -1)
        % to[R=$R_{4}$] (4.5, -1)
        to[short] (4, -1)
        to[short] (4, -2)
        to[short] (3, -2)
        to[short] (3, -2.5) |- (3, -1.5)
        to[R, l_=$R_{5}$] (1, -1.5)
        to[short] (1, -2) -- (0, -2) -- (0, 0);
    
    \draw (1, -2) to[short] (1, -2.5)
        to[R, l_=$R_{6}$] (3, -2.5);
    
    % \draw (3.5, 3.5) to[R, l_=$R_{2}$] (3.5, -1) -- (4, -1);

    % \draw [red, thick] (3.2, 3.75) rectangle (4.8, -1.25);
\end{circuitikz}
\end{figure}    
\end{frame}
\begin{frame}
\frametitle{La siguiente resistencia equivalente}
Ahora tenemos en el circuito las resistencias $R_{5}$ y $R_{6}$ en paralelo, al resolverlas, tendremos la resistencia equivalente $R_{E3}$.
\end{frame}
\begin{frame}
\frametitle{Encontrando $R_{3}$}
\begin{eqnarray*}
\begin{aligned}
\dfrac{1}{R_{E3}} &= \dfrac{1}{R_{5}} + \dfrac{1}{R_{6}} \\[1em] \pause
R_{E3} &= \dfrac{1}{\dfrac{1}{R_{5}} + \dfrac{1}{R_{6}}} = \pause \dfrac{1}{\dfrac{1}{\SI{10}{\ohm}} + \dfrac{1}{\SI{10}{\ohm}}} = \pause \dfrac{1}{\dfrac{1}{5} \si{\ohm}} \\[1em] \pause
R_{E3} &= \SI{5}{\ohm}
\end{aligned}
\end{eqnarray*}
\end{frame}
\begin{frame}[plain]
\begin{figure}
\centering
\begin{circuitikz}
\draw (0, 0) to[battery1={10 V}] (0, 4) % La fuente de voltaje
    to[R=$R_{1}$] (4, 4)
    % to[short] (4, 3.5)
    % to[short] (3.5, 3.5) |- (4.5, 3.5)
    to[short] (4, 3)
    to[R=$R_{E2}$] (4, -1)
    % to[R=$R_{4}$] (4.5, -1)
    to[short] (4, -1)
    to[short] (4, -2)
    to[short] (3, -2)
    % to[short] (3, -2.5) |- (3, -1.5)
    to[R, l_=$R_{E3}$] (1, -2)
    to[short] (0, -2) -- (0, 0) ;

% \draw (1, -2) to[short] (1, -2.5)
%     to[R, l_=$R_{6}$] (3, -2.5);

% \draw (3.5, 3.5) to[R, l_=$R_{2}$] (3.5, -1) -- (4, -1);

% \draw [red, thick] (3.2, 3.75) rectangle (4.8, -1.25);
\end{circuitikz}
\end{figure}    
\end{frame}
\begin{frame}
\frametitle{Resistencia total}
Hemos llegado a tres resistencias conectadas en serie, que al resolverlas, tendremos la resistencia total del circuito $R_{T}$.
\end{frame}
\begin{frame}
\frametitle{Resistencia total}
\begin{figure}
\centering
\begin{circuitikz}
\draw (0, 0) to[battery1={10 V}] (0, 4) % La fuente de voltaje
    to[short] (3, 4)
    to[R=$R_{T}$] (3, 0) -- (0, 0);
\end{circuitikz}
\end{figure}    
\end{frame}

\begin{frame}
\frametitle{Resistencia total}
\begin{align*}
R_{T} &= R_{1} + R_{E2} + R_{E3} = \\[1em]
R_{T} &= \SI{10}{\ohm} + \SI{6.6}{\ohm} + \SI{5}{\ohm} = \\[0.5em]
R_{T} &= \SI{21.6}{\ohm}
\end{align*}
\end{frame}
\begin{frame}
\frametitle{Siguiente paso}
Ocupamos la ley de Ohm para resolver la corriente en la resistencia total:
\pause
\begin{align*}
I_{T} = \dfrac{V}{R_{T}} = \dfrac{\SI{10}{\volt}}{\SI{21.6}{\ohm}} = \SI{0.462}{\ampere}
\end{align*}
\end{frame}
\begin{frame}
\frametitle{Calculando valores en las resistencias}
El procedimiento a seguir será de manera inversa y considerando la configuración que tengamos, \pause es decir, \pause ahora vemos el circuito con tres resistencias en serie:
\end{frame}
\begin{frame}[plain]
\begin{figure}
\centering
\begin{circuitikz}
\draw (0, 0) to[battery1={10 V}] (0, 4) % La fuente de voltaje
    to[R=$R_{1}$] (4, 4)
    to[short] (4, 3)
    to[R=$R_{E2}$] (4, -1)
    to[short] (4, -1)
    to[short] (4, -2)
    to[short] (3, -2)
    to[R, l_=$R_{E3}$] (1, -2)
    to[short] (0, -2) -- (0, 0) ;
\end{circuitikz}
\end{figure}    
\end{frame}
\begin{frame}
\frametitle{Valor de voltaje}
Sabemos que en un circuito en serie con tres resistencias:
\pause
\begin{align*}
V_{T} &= V_{1} + V_{E2} + V_{E3} \\[1em]
I_{T} &= I_{1} = I_{E1} = I_{E2}
\end{align*}
\end{frame}
\begin{frame}
\frametitle{Valores de corriente}
\begin{table}[H]
\centering
\begin{tabular}{c | c }
Resistencia & Corriente \\ \hline
$R_{1}$ & \SI{0.462}{\ampere} \\ \hline
$R_{E1}$ & \SI{0.462}{\ampere} \\ \hline
$R_{E2}$ & \SI{0.462}{\ampere} \\ \hline
\end{tabular}
\end{table}
\end{frame}
\begin{frame}
\frametitle{Valores de voltaje}
Con la ley de Ohm, calculamos el valor de voltaje para cada una de estas resistencias:
\end{frame}
\begin{frame}
\frametitle{Valores de voltaje}
\begin{table}[H]
\centering
\begin{tabular}{c | c | c}
Resistencia & Expresión & Voltaje \\ \hline
$R_{1}$ & $V_{1} = (\SI{0.462}{\ampere})(\SI{10}{\ohm})$ & \SI{4.62}{\volt} \\ \hline
$R_{E2}$ & $V_{E2} = (\SI{0.462}{\ampere})(\SI{6.6}{\ohm})$ & \SI{3.049}{\volt} \\ \hline
$R_{E3}$ & $V_{E3} = (\SI{0.462}{\ampere})(\SI{5}{\ohm})$ & \SI{2.31}{\volt} \\ \hline
\end{tabular}
\end{table}    
\end{frame}
\begin{frame}
\frametitle{Resolviendo $R_{E2}$}
Como ya se tiene el voltaje de la resistencia equivalente $R_{E2}$, \pause podemos obtener los valores de corriente y voltaje de las resistencias $R_{2}$ y $R_{E1}$ que están en paralelo.
\pause
\begin{align*}
V_{E2} = V_{2} = V_{E1}
\end{align*}
\end{frame}
\begin{frame}
\frametitle{Valores de voltaje}
\begin{table}[H]
\centering
\begin{tabular}{c | c }
Resistencia & Voltaje \\ \hline
$R_{2}$ & $V_{2} = \SI{3.049}{\volt}$ \\ \hline
$R_{E1}$ & $V_{E1} = \SI{3.049}{\volt}$ \\ \hline
\end{tabular}
\end{table}    
\end{frame}
\begin{frame}
\frametitle{Valores de corriente}
Obtenemos la corriente usando la ley de Ohm:
\begin{table}[H]
\centering
\begin{tabular}{c | c | c}
Resistencia & Expresión & Corriente \\ \hline
$R_{2}$ & $I_{2} = \dfrac{\SI{3.049}{\volt}}{\SI{10}{\ohm}}$ & \SI{0.304}{\ampere} \\ \hline
$R_{E1}$ & $I_{E1} = \dfrac{\SI{3.049}{\volt}}{\SI{6.6}{\ohm}}$ & \SI{0.1524}{\ampere} \\ \hline
\end{tabular}
\end{table}    
\end{frame}
\begin{frame}
\frametitle{Resolviendo $R_{E1}$}
La resistencia $R_{E1}$ corresponde a las resistencias en serie $R_{3}$ y $R_{4}$, por lo que posible obtener sus valores de corriente y voltaje.
\end{frame}
\begin{frame}
\frametitle{Valores de corriente}
\vspace*{-1cm}
\[ I_{E1} = I_{3} = I_{4} \]
\begin{table}[H]
\centering
\begin{tabular}{c | c }
Resistencia & Corriente \\ \hline
$R_{3}$ & $I_{3} = \SI{0.4619}{\ampere}$ \\ \hline
$R_{4}$ & $I_{4} = \SI{0.4619}{\ampere}$ \\ \hline
\end{tabular}
\end{table}    
\end{frame}
\begin{frame}
\frametitle{Valores de voltaje}
\begin{table}[H]
\centering
\begin{tabular}{c | c | c}
Resistencia & Expresión & Voltaje \\ \hline
$R_{3}$ & $V_{3} = (\SI{0.1524}{\ampere})(\SI{10}{\ohm})$ & \SI{01.524}{\volt} \\ \hline
$R_{4}$ & $V_{4} = (\SI{0.1524}{\ampere})(\SI{10}{\ohm})$ & \SI{01.524}{\volt} \\ \hline
\end{tabular}
\end{table}    
\end{frame}
\begin{frame}
\frametitle{Valores en $R_{E3}$}
Nos resta calcular los valores de corriente y voltaje en $R_{E3}$, que la conforman las resistencias $R_{5}$ y $R_{6}$ conectadas en paralelo.
\end{frame}
\begin{frame}
\frametitle{Valores de voltaje}
\vspace*{-1cm}
\[ V_{E3} = V_{5} = V_{6} \]
\begin{table}[H]
\centering
\begin{tabular}{c | c }
Resistencia & Corriente \\ \hline
$R_{5}$ & $V_{5} = \SI{2.31}{\volt}$ \\ \hline
$R_{6}$ & $V_{6} = \SI{2.31}{\volt}$ \\ \hline
\end{tabular}
\end{table}    
\end{frame}
\begin{frame}
\frametitle{Valores de corriente}
Obtenemos la corriente usando la ley de Ohm:
\begin{table}[H]
\centering
\begin{tabular}{c | c | c}
Resistencia & Expresión & Corriente \\ \hline
$R_{5}$ & $I_{5} = \dfrac{\SI{2.31}{\volt}}{\SI{10}{\ohm}}$ & \SI{0.231}{\ampere} \\ \hline
$R_{6}$ & $I_{6} = \dfrac{\SI{2.31}{\volt}}{\SI{10}{\ohm}}$ & \SI{0.231}{\ampere} \\ \hline
\end{tabular}
\end{table}    
\end{frame}
\begin{frame}
\frametitle{Trabajo completo}
Ya se han calculado las corrientes y voltajes en cada una de las resistencias del circuito mixto, se aconseja presentar una tabla con los valores.
\end{frame}
\begin{frame}
\frametitle{Valores para las resistencias}
\begin{table}[H]
\centering
\renewcommand{\arraystretch}{0.8}
\begin{tabular}{| c | c | c |} \hline
Resistencia & Voltaje & Corriente \\ \hline
$R_{1}$ & \SI{4.620}{\volt} & \SI{0.462}{\ampere} \\ \hline
$R_{2}$ & \SI{3.049}{\volt} & \SI{0.304}{\ampere} \\ \hline
$R_{3}$ & \SI{1.524}{\volt} & \SI{0.1524}{\ampere} \\ \hline
$R_{4}$ & \SI{1.524}{\volt} & \SI{0.1524}{\ampere} \\ \hline
$R_{5}$ & \SI{2.31}{\volt} & \SI{0.231}{\ampere} \\ \hline
$R_{6}$ & \SI{2.31}{\volt} & \SI{0.231}{\ampere} \\ \hline
\end{tabular}
\end{table}
\end{frame}
\end{document}