\documentclass[14pt]{beamer}
\usepackage{./Estilos/BeamerUVM}
\usepackage{./Estilos/ColoresLatex}
\usetheme{Madrid}
\usecolortheme{default}
%\useoutertheme{default}
\setbeamercovered{invisible}
% or whatever (possibly just delete it)
\setbeamertemplate{section in toc}[sections numbered]
\setbeamertemplate{subsection in toc}[subsections numbered]
\setbeamertemplate{subsection in toc}{\leavevmode\leftskip=3.2em\rlap{\hskip-2em\inserttocsectionnumber.\inserttocsubsectionnumber}\inserttocsubsection\par}
% \setbeamercolor{section in toc}{fg=blue}
% \setbeamercolor{subsection in toc}{fg=blue}
% \setbeamercolor{frametitle}{fg=blue}
\setbeamertemplate{caption}[numbered]

\setbeamertemplate{footline}
\beamertemplatenavigationsymbolsempty
\setbeamertemplate{headline}{}


\makeatletter
% \setbeamercolor{section in foot}{bg=gray!30, fg=black!90!orange}
% \setbeamercolor{subsection in foot}{bg=blue!30}
% \setbeamercolor{date in foot}{bg=black}
\setbeamertemplate{footline}
{
  \leavevmode%
  \hbox{%
  \begin{beamercolorbox}[wd=.333333\paperwidth,ht=2.25ex,dp=1ex,center]{section in foot}%
    \usebeamerfont{section in foot} {\insertsection}
  \end{beamercolorbox}%
  \begin{beamercolorbox}[wd=.333333\paperwidth,ht=2.25ex,dp=1ex,center]{subsection in foot}%
    \usebeamerfont{subsection in foot}  \insertsubsection
  \end{beamercolorbox}%
  \begin{beamercolorbox}[wd=.333333\paperwidth,ht=2.25ex,dp=1ex,right]{date in head/foot}%
    \usebeamerfont{date in head/foot} \insertshortdate{} \hspace*{2em}
    \insertframenumber{} / \inserttotalframenumber \hspace*{2ex} 
  \end{beamercolorbox}}%
  \vskip0pt%
}
\makeatother

\makeatletter
\patchcmd{\beamer@sectionintoc}{\vskip1.5em}{\vskip0.8em}{}{}
\makeatother

% \usefonttheme{serif}
\usepackage[clock]{ifsym}

\sisetup{per-mode=symbol}
\resetcounteronoverlays{saveenumi}

\title{\Large{Presentación syllabus - Laboratorio} \\ \normalsize{Física IV (Área II)}}
\date{29 de agosto de 2023}

\begin{document}
\maketitle

\section*{Contenido}
\frame{\frametitle{Contenido} \tableofcontents[currentsection, hideallsubsections]}

\section{Objetivos}
\frame{\tableofcontents[currentsection, hideothersubsections]}
\subsection{Metas esperadas}

\begin{frame}
\frametitle{Objetivos del curso}
El alumno aplicará los conceptos, principios, leyes, lenguajes de representación y metodologías de la Física a partir de la comprensión y explicación de fenómenos físicos inherentes en procesos químicos y biológicos específicos, 
\end{frame}
\begin{frame}
\frametitle{Objetivos del curso}
Con el fin de que emplee los instrumentos tecnológicos de punta de manera razonada (inductivo, deductivo y abductivo) y argumentada científicamente, así como con una actitud responsable y propositiva.
\end{frame}

\section{Evaluación}
\frame{\tableofcontents[currentsection, hideothersubsections]}
\subsection{Esquema de evaluación}

\begin{frame}
\frametitle{Trabajo en el Laboratorio}
El trabajo en Laboratorio consistirá en el montaje de prácticas dirigidas en las sesiones de clase.
\end{frame}
\begin{frame}
\frametitle{Etapas de trabajo}
\setbeamercolor{item projected}{bg=aquamarine,fg=black}
\setbeamertemplate{enumerate items}{%
\usebeamercolor[bg]{item projected}%
\raisebox{1.5pt}{\colorbox{bg}{\color{fg}\footnotesize\insertenumlabel}}%
}
\begin{enumerate}[<+->]
\item Discusión de la práctica.
\item Montaje experimental.
\item Interpretación de resultados y reporte.
\end{enumerate}
\end{frame}
\begin{frame}
\frametitle{Discusión de la práctica}
Durante el año escolar se realizarán varias prácticas, en una primera sesión se discutirá sobre el objetivo de la misma, así como el marco teórico para su comprensión.
\end{frame}
\begin{frame}
\frametitle{Discusión de la práctica}
El alumno deberá de complementar la revisión durante la semana, de tal manera que en la clase de montaje, tendrá el soporte de conocimiento necesario para realizar la práctica.
\end{frame}
\begin{frame}
\frametitle{Montaje de la práctica}
En una segunda sesión se realizará el montaje de la práctica durante la clase, esto implica trabajo en equipo.
\end{frame}
\begin{frame}
\frametitle{Recabando datos e información}
Cada equipo de trabajo, deberá de recolectar los datos de la práctica, de tal manera que con trabajo adicional durante la semana, llegará a la siguiente sesión con un trabajo preliminar.
\end{frame}
\begin{frame}
\frametitle{Interpretación de datos y reporte}
En la tercera sesión, cada equipo de trabajo revisará los resultados preliminares, discutirán sobre los hechos hallados y revisarán su congruencia con el marco téorico.
\end{frame}
\begin{frame}
\frametitle{Elaboración del reporte}
Una vez revisada la parte de interpretación, cada alumno realizará un reporte de la práctica.
\\
\bigskip
\pause
Se dispondrá de una rúbrica para la evaluación del reporte de la práctica.
\end{frame}
\begin{frame}
\frametitle{Calificación de Laboratorio}
El peso de la calificación de Laboratorio corresponde al $30 \%$ de la calificación de cada examen parcial.
\\
\bigskip
\pause
Se tendrán cuatro exámenes parciales durante el ciclo escolar.
\end{frame}

\section{Temas para las prácticas}
\frame{\tableofcontents[currentsection, hideothersubsections]}
\subsection{Física de la visión y audición}

\begin{frame}
\frametitle{Prácticas de audición y visión}
Como primer tema para el trabajo en Laboratorio, se realizarán prácticas correspondientes a los temas de audición y visión:
\end{frame}
\begin{frame}
\frametitle{Prácticas a realizar}
\setbeamercolor{item projected}{bg=darkred,fg=white}
\setbeamertemplate{enumerate items}{%
\usebeamercolor[bg]{item projected}%
\raisebox{1.5pt}{\colorbox{bg}{\color{fg}\footnotesize\insertenumlabel}}%
}
\begin{enumerate}[<+->]
\item Oído y transferencia de energía.
\item Efecto Doppler.
\item Ondas electromagnéticas (luz visible)
\item Principio de Huygens.
\seti
\end{enumerate}
\end{frame}
\begin{frame}
\frametitle{Prácticas a realizar}
\setbeamercolor{item projected}{bg=darkred,fg=white}
\setbeamertemplate{enumerate items}{%
\usebeamercolor[bg]{item projected}%
\raisebox{1.5pt}{\colorbox{bg}{\color{fg}\footnotesize\insertenumlabel}}%
}
\begin{enumerate}[<+->]
\conti
\item Óptica geométrica.
\item Refracción (índice de refracción, ley de Snell).
\item Lentes delgadas.
\item Formación de imágenes en lentes.
\end{enumerate}
\end{frame}

\subsection{Fluidos y pulsos eléctricos}

\begin{frame}
\frametitle{Prácticas para la segunda parte}
Como tema para la segunda parte del curso, se revisarán actividades de fluidos y pulsos eléctricos en el cuerpo humano.
\end{frame}
\begin{frame}
\frametitle{Prácticas para la segunda parte}
\setbeamercolor{item projected}{bg=darkscarlet,fg=white}
\setbeamertemplate{enumerate items}{%
\usebeamercolor[bg]{item projected}%
\raisebox{1.5pt}{\colorbox{bg}{\color{fg}\footnotesize\insertenumlabel}}%
}
\begin{enumerate}[<+->]
\item Ecuación de continuidad.
\item Ecuación de Bernoulli.
\item Ecuación de Poiseuille.
\seti
\end{enumerate}
\end{frame}
\begin{frame}
\frametitle{Prácticas para la segunda parte}
\setbeamercolor{item projected}{bg=darkscarlet,fg=white}
\setbeamertemplate{enumerate items}{%
\usebeamercolor[bg]{item projected}%
\raisebox{1.5pt}{\colorbox{bg}{\color{fg}\footnotesize\insertenumlabel}}%
}
\begin{enumerate}[<+->]
\conti
\item Corriente directa y alterna.
\item Ley de Ohm.
\item Circuitos eléctricos. Mixtos, RC y RCL.
\item Impedancia eléctrica.
\item Potencial de acción.
\end{enumerate}
\end{frame}

\section{Trabajo en Laboratorio}
\frame{\tableofcontents[currentsection, hideothersubsections]}

\subsection{Asistencia}

\begin{frame}
\frametitle{Pase de asistencia}
Se realizará el \textocolor{red}{pase de asistencia} luego de la tolerancia para ingresar a la clase, que es de 5 minutos.
\end{frame}
\begin{frame}
\frametitle{Pase de asistencia}
En caso de que la(el) alumna(o) no se encuentre en la clase al momento del pase de asistencia, se le marcará como inasistencia, así haya ingresado luego de completar el pase.
\end{frame}
\begin{frame}
\frametitle{De la asistencia}
La sesión de Laboratorio se lleva a cabo una vez a la semana.
\\
\bigskip
\pause
Siendo indispensable asistir a cada una de las clases de Laboratorio.
\end{frame}
    
\subsection{Elementos necesarios}

\begin{frame}
\frametitle{Elementos necesarios}
Para el trabajo en laboratorio, es necesario utilizar una BATA BLANCA.
\\
\bigskip
\pause
Revisaremos que en el Manual de Seguridad y Normas de Higiene el Laboratorio, el uso de Bata es requisito.
\end{frame}
\begin{frame}
\frametitle{¿Qué pasa si no traigo la bata?}
En caso de que no se tenga la bata, la(el) alumna(o) no podrá permanecer en el Laboratorio, \pause se notificará a la Coordinación Académica para que los acompañe a la Biblioteca en donde tendrán que realizar una actividad entregable al concluir la clase.
\end{frame}
\begin{frame}
\frametitle{¿Qué pasa si no traigo la bata?}
Deberán de completar la actividad, pero no podrán reponer la actividad de Laboratorio y contarán con inasistencia.
\\
\bigskip
\pause
Es recomendable que se prevengan para contar con una bata blanca.
\end{frame}

\begin{frame}
\frametitle{El Reglamento}
Revisaremos en la siguiente sesión el Manual de Seguridad y las Normas de Higiene para el Laboratorio.
\end{frame}

\subsection{Primera actividad}

\begin{frame}
\frametitle{Primera actividad}
Será necesario que cada alumno incluya en su cuardeno, carpeta de notas, etc. la \textocolor{ao}{síntesis de la asignatura}, con la firma del alumno así como del Tutor.
\end{frame}
\begin{frame}
\frametitle{Primera actividad}
De esta manera, se tendrá la claridad en el trabajo, la didáctica en la clase, el reglamento y las actividades a realizar durante el año escolar.
\end{frame}
\begin{frame}
\frametitle{Actividad que cuenta}
El presentar la síntesis de la asignatura debidamente firmada, contará en el puntaje de evaluación para el Laboratorio.
\end{frame}
\begin{frame}
\frametitle{Entrega de la actividad}
Se abrirá una asignación en Teams para el envío de una foto de la síntesis firmada por el alumno y el Tutor, además de tenerla en su cuaderno de trabajo.
\end{frame}
\begin{frame}
\frametitle{Primeras actividades}
Se informa al grupo que se está acabando de afinar el espacio de Laboratorio para las clases, \pause por lo que las primeras tres actividades se llevarán a cabo en el aula.
\end{frame}
\begin{frame}
\frametitle{Primeras actividades}
Estas primeras actividades contabilizarán como prácticas y por lo tanto, sumarán puntos para el primer examen parcial.
\end{frame}
\begin{frame}
\frametitle{Consideración importante}
Como ya se mencionó, se tendrá una clase de Laboratorio a la semana.
\\
\bigskip
\pause
No asistir a una clase compromete la evaluación, ya que no se podrá reponer la actividad a la que se haya asistido.
\end{frame}
\end{document}