\documentclass[14pt]{extarticle}
\usepackage[utf8]{inputenc}
\usepackage[T1]{fontenc}
\usepackage[spanish,es-lcroman]{babel}
\usepackage{amsmath}
\usepackage{amsthm}
\usepackage{physics}
\usepackage{tikz}
\usepackage{float}
\usepackage[autostyle,spanish=mexican]{csquotes}
\usepackage[per-mode=symbol]{siunitx}
\usepackage{gensymb}
\usepackage{multicol}
\usepackage{enumitem}
\usepackage[left=2.00cm, right=2.00cm, top=2.00cm, 
     bottom=2.00cm]{geometry}

%\renewcommand{\questionlabel}{\thequestion)}
\decimalpoint
\sisetup{bracket-numbers = false}

\title{\vspace*{-2cm} Práctica 1 - Análisis preliminar \\  Física IV (Área II) \vspace{-5ex}}
\date{}

\begin{document}
\maketitle

Luego de haber realizado el montaje experimental para las ondas estacionarias, deberás de preparar un análisis preliminar de los datos experimentales, con los siguientes puntos:

\begin{enumerate}
\item En la tabla de datos experimentales, obtén la longitud de onda (en metros) para cada registro:
\begin{table}[H]
\centering
\begin{tabular}{c | c | c}
$L$ [\unit{\meter}] & Bucles & $\lambda$ [\unit{\meter}] \\ \hline
\vdots & \vdots & \vdots \\ \hline
\end{tabular}
\end{table}
Para obtener la longitud de onda, recuerda la expresión que vimos en clase:
\begin{align*}
\lambda = \dfrac{2 \, L}{n}
\end{align*}
donde $L$ es la longitud del hilo y $n$ es el número de bucles.
\item Responde las siguientes preguntas:
\begin{enumerate}
\item ¿Cómo explicas que el número de bucles que se generaron, no siguen un patrón de proporción? Es decir, que a mayor longitud se espararían más bucles, revisa que hay longitudes donde $n$ no cambia o incluso es menor.
\item Fue claro el notar que el hilo cáñamo que se utilizó se enredó, ¿por qué se enredó el hilo?
\item ¿Cómo mejorarías la práctica?
\item Si te piden medir la amplitud de la onda estacionaria, ¿cómo medirías ese valor?
\item ¿Se cumplió el objetivo de la práctica?
\item Las hipótesis que se plantearon, ¿son correctas?
\end{enumerate}
\item Una vez que hayas respondido todo lo anterior, deberás de anotarlo en tu archivo donde tienes el marco teórico que ya presentaste para la práctica 1.
\item Deberás de enviar tu archivo actualizado en la asignación por Teams.
\item En la siguiente clase se te pedirá que expongas tu análisis preliminar, por lo que deberás de preparar la información solicitada.
\item En caso de que incluyas imágenes deberás de adjuntarlas en la asignación.
\item Recuerda que esta actividad cuenta para la calificación.
\item El trabajo de análisis es INDIVIDUAL.
\end{enumerate}

\end{document}