\documentclass[14pt]{extarticle}
\usepackage[utf8]{inputenc}
\usepackage[T1]{fontenc}
\usepackage[spanish,es-lcroman]{babel}
\usepackage{amsmath}
\usepackage{amsthm}
\usepackage{physics}
\usepackage{tikz}
\usepackage{float}
\usepackage[autostyle,spanish=mexican]{csquotes}
\usepackage[per-mode=symbol]{siunitx}
\usepackage{gensymb}
\usepackage{multicol}
\usepackage{enumitem}
\usepackage[left=2.00cm, right=2.00cm, top=2.00cm, 
     bottom=2.00cm]{geometry}
\usepackage{Estilos/ColoresLatex}
\usepackage{makecell}

\newcommand{\textocolor}[2]{\textbf{\textcolor{#1}{#2}}}
\sisetup{per-mode=symbol}
\DeclareSIUnit[number-unit-product = {\,}]\cal{cal}
\DeclareSIUnit{\dB}{dB}
%\renewcommand{\questionlabel}{\thequestion)}
\decimalpoint
\sisetup{bracket-numbers = false}

\title{\vspace*{-2cm} Actividad 2 - Ondas sonoras \\ Solución - Física IV (Área II) \vspace{-5ex}}
\date{}

\begin{document}
\maketitle

\section*{Ejercicios a resolver.}

% Tippens Sec. 25.5 Ondas sonoras audibles.
\begin{enumerate}
\item ¿Cuál es el nivel de intensidad en decibeles de un sonido que tiene una intensidad de \SI{4d-5}{\watt\per\square\meter}?

\begin{minipage}[t]{0.4\linewidth}
\textocolor{red}{Datos:}
\begin{align*}
I &= \SI{4d-5}{\watt\per\square\meter} \\[0.5em]
I_{0} &= \SI{1d-12}{\watt\per\square\meter}
\end{align*}
\end{minipage}
\hspace{0.4cm}
\begin{minipage}[t]{0.4\linewidth}
\textocolor{red}{Expresión:}
\begin{align*}
B = 10 \, \log \left( \dfrac{I}{I_{0}} \right)
\end{align*}
\end{minipage}

\textocolor{red}{Sustitución:}
\begin{align*}
B &= 10 \, \log \left( \dfrac{\SI{4d-5}{\watt\per\square\meter}}{\SI{1d-12}{\watt\per\square\meter}} \right) = 10 \, \log \left( \num{4d7} \right) = \\[0.5em]
B &= 10 \, (7.60) = \SI{76}{\dB}
\end{align*}
\item La intensidad de un sonido es \SI{6d-8}{\watt\per\square\meter} ¿Cuál es el nivel de intensidad?

\begin{minipage}[t]{0.4\linewidth}
\textocolor{red}{Datos:}
\begin{align*}
I &= \SI{6d-8}{\watt\per\square\meter} \\[0.5em]
I_{0} &= \SI{1d-12}{\watt\per\square\meter}
\end{align*}
\end{minipage}
\hspace{0.4cm}
\begin{minipage}[t]{0.4\linewidth}
\textocolor{red}{Expresión:}
\begin{align*}
B = 10 \, \log \left( \dfrac{I}{I_{0}} \right)
\end{align*}
\end{minipage}

\textocolor{red}{Sustitución:}
\begin{align*}
B &= 10 \, \log \left( \dfrac{\SI{6d-8}{\watt\per\square\meter}}{\SI{1d-12}{\watt\per\square\meter}} \right) = 10 \, \log \left( \num{6d4} \right) = \\[0.5em]
B &= 10 \, (4.778) = \SI{47.78}{\dB}
\end{align*}
\item A cierta distancia de un silbato se mide un sonido de \SI{60}{\dB}. ¿Cuál es la intensidad de ese sonido en \unit{\watt\per\square\meter}?

\begin{minipage}[t]{0.4\linewidth}
\textocolor{red}{Datos:}
\begin{align*}
B &= \SI{60}{\dB} \\[0.5em]
I_{0} &= \SI{1d-12}{\watt\per\square\meter}
\end{align*}
\end{minipage}
\hspace{0.4cm}
\begin{minipage}[t]{0.4\linewidth}
\textocolor{red}{Expresión:}
\begin{align*}
B &= 10 \, \log \left( \dfrac{I}{I_{0}} \right) \\[0.5em]
\log \left( \dfrac{I}{I_{0}} \right) &= \dfrac{B}{10} \\[0.5em]
\dfrac{I}{I_{0}} &= 10^{(B/10)} \\[0.5em]
I &= I_{0} \, 10^{(B/10)}
\end{align*}
\end{minipage}

\textocolor{red}{Sustitución:}
\begin{align*}
I &= (\SI{1d-12}{\watt\per\square\meter}) \left( 10^{6} \right) = \SI{1d-6}{\watt\per\square\meter}
\end{align*}
\item ¿Cuál es la intensidad de un sonido de \SI{250}{\dB}?

\begin{minipage}[t]{0.4\linewidth}
\textocolor{red}{Datos:}
\begin{align*}
B &= \SI{250}{\dB} \\[0.5em]
I_{0} &= \SI{1d-12}{\watt\per\square\meter}
\end{align*}
\end{minipage}
\hspace{0.4cm}
\begin{minipage}[t]{0.4\linewidth}
\textocolor{red}{Expresión:}
\begin{align*}
I &= I_{0} \, 10^{(B/10)}
\end{align*}
\end{minipage}

\textocolor{red}{Sustitución:}
\begin{align*}
I &= (\SI{1d-12}{\watt\per\square\meter}) \left( 10^{25} \right) = \SI{1d13}{\watt\per\square\meter}
\end{align*}
\item Calcula las intensidades correspondientes a sonidos de \num{10}, \num{20} y \SI{30}{\dB}.

\begin{minipage}[t]{0.4\linewidth}
\textocolor{red}{Datos:}
\begin{align*}
B_{1} &= \SI{10}{\dB} \\[0.5em]
B_{2} &= \SI{20}{\dB} \\[0.5em]
B_{3} &= \SI{30}{\dB} \\[0.5em]
I_{0} &= \SI{1d-12}{\watt\per\square\meter}
\end{align*}
\end{minipage}
\hspace{0.4cm}
\begin{minipage}[t]{0.4\linewidth}
\textocolor{red}{Expresión:}
\begin{align*}
I_{i} &= I_{0} \, 10^{(B_{i}/10)} \hspace{1cm} i = 1, 2, 3
\end{align*}
\end{minipage}

\textocolor{red}{Sustitución:}
\begin{align*}
I_{1} &= (\SI{1d-12}{\watt\per\square\meter}) \left( 10 \right) = \SI{1d-11}{\watt\per\square\meter} \\[0.5em]
I_{2} &= (\SI{1d-12}{\watt\per\square\meter}) \left( 100 \right) = \SI{1d-10}{\watt\per\square\meter} \\[0.5em]
I_{3} &= (\SI{1d-12}{\watt\per\square\meter}) \left( 1000 \right) = \SI{1d-9}{\watt\per\square\meter}
\end{align*}
\item Calcula los niveles de intensidad que corresponden a sonidos de \SI{1d-6}{\watt\per\square\meter}, \SI{25d-6}{\watt\per\square\meter} y \SI{300d-6}{\watt\per\square\meter}.

\begin{minipage}[t]{0.4\linewidth}
\textocolor{red}{Datos:}
\begin{align*}
I_{1} &= \SI{1d-6}{\watt\per\square\meter} \\[0.5em]
I_{2} &= \SI{25d-6}{\watt\per\square\meter} \\[0.5em]
I_{3} &= \SI{300d-6}{\watt\per\square\meter} \\[0.5em]
I_{0} &= \SI{1d-12}{\watt\per\square\meter}
\end{align*}
\end{minipage}
\hspace{0.4cm}
\begin{minipage}[t]{0.4\linewidth}
\textocolor{red}{Expresión:}
\begin{align*}
B_{i} = 10 \, \log \left( \dfrac{I_{i}}{I_{0}} \right) \hspace{0.5cm} i = 1, 2, 3
\end{align*}
\end{minipage}

\textocolor{red}{Sustitución:}
\begin{align*}
B_{1} &= 10 \, \log \left( \dfrac{\SI{1d-6}{\watt\per\square\meter}}{\SI{1d-12}{\watt\per\square\meter}} \right) = 10 \, \log (\num{d6}) = 10 \, (6) = \SI{60}{\dB} \\[0.5em]
B_{2} &= 10 \, \log \left( \dfrac{\SI{25d-6}{\watt\per\square\meter}}{\SI{1d-12}{\watt\per\square\meter}} \right) = 10 \, \log (\num{2.5d7}) = 10 \, (7.397) = \\[0.5em]
B_{2} &= \SI{73.97}{\dB} \\[0.5em]
B_{3} &= 10 \, \log \left( \dfrac{\SI{300d-6}{\watt\per\square\meter}}{\SI{1d-12}{\watt\per\square\meter}} \right) = 10 \, \log (\num{3d8}) = \\[0.5em]
B_{3} &= 10 \, (8.477) = \SI{84.77}{\dB}
\end{align*}    
\end{enumerate}

\end{document}