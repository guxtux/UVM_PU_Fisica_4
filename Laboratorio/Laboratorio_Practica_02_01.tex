\documentclass[14pt]{extarticle}
\usepackage[utf8]{inputenc}
\usepackage[T1]{fontenc}
\usepackage[spanish,es-lcroman]{babel}
\usepackage{amsmath}
\usepackage{amsthm}
\usepackage{physics}
\usepackage{tikz}
\usepackage{float}
\usepackage[autostyle,spanish=mexican]{csquotes}
\usepackage[per-mode=symbol]{siunitx}
\usepackage{gensymb}
\usepackage{multicol}
\usepackage{enumitem}
\usepackage[left=2.00cm, right=2.00cm, top=2.00cm, 
     bottom=2.00cm]{geometry}

%\renewcommand{\questionlabel}{\thequestion)}
\decimalpoint
\sisetup{bracket-numbers = false}

\title{\vspace*{-2cm} Práctica 2 - Sonido y distancia \\  Física IV (Área II) \vspace{-5ex}}
\date{}

\begin{document}
\maketitle

\section{Datos para la práctica.}

\begin{itemize}
\itemsep0em 
\item  \textbf{Práctica:} 2.
\item \textbf{Unidad:} Uno
\item \textbf{Temática:} Intensidad de las ondas sonoras.
\item \textbf{Nombre de la práctica:} Sonido y distancia.
\item \textbf{Número de sesiones que se requieren para la práctica:} Dos.
\end{itemize}
\textbf{Planteamiento del problema:} Se ha definido el sonido como una onda mecánica longitudinal que se propaga por un medio elástico. Ésta es una definición amplia que no impone restricciones a ninguna frecuencia del sonido. Desde la perspectiva biofísica, hay un interés promordial en las ondas sonoras que tienen la capacidad de afectar el sentido del oído.

Cuando se estudian los sonidos audibles, nuevamente en la biofísica se usan los términos fuerza, tono y timbre para describir las sensaciones producidas. Es importante destacar que estos términos representan magnitudes sensoriales y, por tanto, subjetivas:
\begin{enumerate}[label=\alph*)]
\itemsep0em 
\item Lo que es volumen fuerte para una persona es moderado para otra.
\item Lo que alguien percibe como calidad, otro lo considera inferior.
\end{enumerate}
Como siempre, en la física se debe trabajar con definiciones explícitas y mensurables.

Las ondas sonoras constituyen un flujo de energía a través de la materia. La intensidad de una onda sonora específica es una medida de la razón en la que la energía se propaga por cierto volumen espacial.

\section{Marco teórico.}

Responde las siguientes preguntas:

\begin{enumerate}
\itemsep0.5em 
\item ¿Cómo se relaciona la intensidad del sonido y la potencia del mismo?
\item ¿De qué factores depende la intensidad de sonido?
\end{enumerate}

\section{Objetivos.}

\subsection{General.}

Determinar la relación entre la intensidad del sonido de una fuente y la distancia en la que se registra.

\subsection{Específicos.}

\begin{enumerate}
\item Medir la intensidad de sonido de una fuente.
\item Obtener con los datos experimentales la relación entre la intensidad de sonido y la distancia en la que se registra.
\end{enumerate}

\section{Hipótesis.}

A mayor distancia de registro, la intensidad del sonido disminuye en relación al cuadrado inverso de la distancia.

\section{Material.}

\begin{enumerate}
\itemsep0.15em 
\item Tubo de cartón de \SI{50}{\centi\meter}.
\item Cinta adhesiva.
\item Flexómetro.
\item Bolsa de papel o de plástico
\end{enumerate}
Para esta actividad, se requiere adicionalmente de: una bocina pequeña inálambrica y el uso del celular con una aplicación que nos permita generar una señal de sonido y otra para registrar el nivel de sonido, es decir, un sonómetro; la aplicación ya debe de estar instalada en el equipo celular. 

\section{Registro de datos.}

\begin{table}[H]
\centering
\begin{tabular}{| c | c | c | c |} \hline
Registro & \unit{dB} & \unit{\meter} & Intensidad (\unit{\watt\per\square\meter}) \\ \hline
1 & & & \\ \hline    
\vdots & & & \\ \hline    
\end{tabular}
\end{table}
\end{document}