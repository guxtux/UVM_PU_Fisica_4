\documentclass[14pt]{extarticle}
\usepackage[utf8]{inputenc}
\usepackage[T1]{fontenc}
\usepackage[spanish,es-lcroman]{babel}
\usepackage{amsmath}
\usepackage{amsthm}
\usepackage{physics}
\usepackage{tikz}
\usepackage{float}
\usepackage[autostyle,spanish=mexican]{csquotes}
\usepackage[per-mode=symbol]{siunitx}
\usepackage{gensymb}
\usepackage{multicol}
\usepackage{enumitem}
\usepackage[left=2.00cm, right=2.00cm, top=2.00cm, 
     bottom=2.00cm]{geometry}

%\renewcommand{\questionlabel}{\thequestion)}
\decimalpoint
\sisetup{bracket-numbers = false}

\title{\vspace*{-2cm} Práctica 2 - Sonido y distancia \\  Análisis preliminar - Física IV (Área II) \vspace{-5ex}}
\date{}

\begin{document}
\maketitle

\section{Con los datos experimentales.}
Una vez registrados los datos experimentales en la tabla que se indicó en la sesión de laboratorio:

Frecuencia utilizada: \rule{4cm}{0.3mm}
\begin{table}[H]
\centering
\begin{tabular}{| c | c |  c |} \hline
Distancia (\unit{\meter}) & Intensidad (\unit{dB}) &  Intensidad (\unit{\watt\per\square\meter}) \\ \hline
$0.1$ & & \\ \hline    
$0.2$ & & \\ \hline    
$0.3$ & & \\ \hline    
\vdots & & \\ \hline    
$1.0$ & & \\ \hline    
$1.5$ & & \\ \hline    
$2.0$ & & \\ \hline    
$2.5$ & & \\ \hline    
$3.0$ & & \\ \hline    
\end{tabular}
\end{table}

\begin{enumerate}
\item Anota el nombre de las aplicaciones que utilizaron para el generador de funciones y el sonómetro, agrega el sistema en el que se utilizaron (Android, iOS)

Generador de funciones: \rule{5cm}{0.3mm} \\
Sonómetro: \rule{5cm}{0.3mm} 
\item Grafica los datos: en el eje de las ordenadas con la intensidad en \unit{\watt\per\square\meter} y en el eje de las abscisas la distancia en \unit{\meter}.
\end{enumerate}
Responde las siguientes preguntas:
\begin{enumerate}[resume*]
\item ¿Se relacionan sus resultados con la hipótesis inicial?
\item ¿Qué sucedió cuando alejaron el sonómetro del generador de funciones?
\item Describe la curva en la gráfica. ¿Se presenta alguna tendencia con los datos experimentales? (Por ejemplo, lineal, cuadrática, logarítmica, etc). Tendrás que investigar en qué consiste cada una de las tendencias; cuando un eje ordenado se modifica a una escala logarítmica se le conoce como gráfica semilog, cuando los dos ejes se cambian a escalas logarítmicas, se le conoce como gráfica log-log. ¿En qué casos se utilizan éstas gráficas?
\end{enumerate}

\section{Discusión de los resultados.}

\begin{enumerate}
\item De acuerdo con los resultados y la información proporcionada en el marco teórico, ¿por qué el nivel de intensidad del sonido disminuye a medida que se aleja la fuente de sonido?
\item Además de la distancia, menciona qué otros factores que influyen en la intensidad del sonido.
\item ¿De qué manera la intensidad del sonido varía con los cambios en la distancia? Explica cómo se llama esta relación y la fórmula matemática que la define.
\item Si la ley del cuadrado inverso aplica con la intensidad del sonido y la distancia, explica cómo es posible que dentro de un auditorio (por ejemplo en un concierto de piano), el sonido se escucha con bastante claridad ya sea en las primeras filas, como en las filas más alejadas del segundo piso de un auditorio.
\item ¿Influyeron en los resultados el tipo de generador de funciones y el sonómetro? Compara los resultados de los otros equipos.
\end{enumerate}

\section{Para la siguiente clase.}

Responde las preguntas ya que te servirán para elaborar tus conclusiones, en la siguiente clase lleva en tu cuaderno la respuestas y la gráfica, ya que expondrás los resultados obtenidos y presentarás tu discusión de los mismos.

\end{document}