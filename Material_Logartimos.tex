\documentclass[14pt]{extarticle}
\usepackage[utf8]{inputenc}
\usepackage[T1]{fontenc}
\usepackage[spanish,es-lcroman]{babel}
\usepackage{amsmath}
\usepackage{amsthm}
\usepackage{physics}
\usepackage{tikz}
\usepackage{float}
\usepackage[autostyle,spanish=mexican]{csquotes}
\usepackage[per-mode=symbol]{siunitx}
\usepackage{gensymb}
\usepackage{multicol}
\usepackage{enumitem}
\usepackage[left=2.00cm, right=2.00cm, top=2.00cm, 
     bottom=2.00cm]{geometry}
\usepackage{Estilos/ColoresLatex}
\usepackage{makecell}

\newcommand{\textocolor}[2]{\textbf{\textcolor{#1}{#2}}}
\sisetup{per-mode=symbol}
\DeclareSIUnit[number-unit-product = {\,}]\cal{cal}
\DeclareSIUnit{\dB}{dB}
%\renewcommand{\questionlabel}{\thequestion)}
\decimalpoint
\sisetup{bracket-numbers = false}

\title{\vspace*{-2cm} Material Logaritmos \\ \large{Física IV (Área II)} \vspace{-5ex}}
\date{}

\begin{document}
\maketitle

Los logaritmos se definen en la forma siguiente:
\begin{align*}
\text{si} \hspace{0.2cm} y = A^{x} \hspace{0.3cm} \text{entonces} \hspace{0.3cm} x = \log_{A} \, y
\end{align*}
Esto es: el logaritmo de un número $y$ a la base $A$ es aquel número que, como exponente de $A$, devuelve el número $y$. Para \textbf{logaritmos comunes}, la base es \num{10}, de modo que
\begin{align*}
\text{si} \hspace{0.2cm} y = 10^{x} \hspace{0.3cm} \text{entonces} \hspace{0.3cm} x = \log \, y
\end{align*}
El subíndice \num{10} en $\log_{10}$ por lo general se omite cuando se trata con logaritmos comunes.
\par
Otra base importante es la base exponencial $e = 2.718 \dots$ un número natural. Tales logaritmos se llaman \textbf{logaritmos naturales (o neperianos)} y se escriben $\ln \, y$. Por tanto,
\begin{align*}
\text{si} \hspace{0.2cm} y = e^{x} \hspace{0.3cm} \text{entonces} \hspace{0.3cm} x = \ln \, y
\end{align*}
Para cualquier número $y$, los dos tipos de logaritmo se relacionan mediante
\begin{align*}
\ln \, y = 2.3026 \, \log \, y
\end{align*}
Algunas reglas simples para logaritmos son las siguientes:
\begin{enumerate}
\item $\log (a \, b) = \log \, a + \log \, b$ 
\item $\log \left( \dfrac{a}{b} \right) = \log \, a - \log \, b$
\item $\log \, a^{n} = n \, \log \, a$
\end{enumerate}
Estas tres reglas se aplican a cualquier tipo de logaritmo.
\end{document}
