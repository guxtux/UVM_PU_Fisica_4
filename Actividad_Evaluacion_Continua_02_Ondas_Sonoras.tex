\documentclass[14pt]{extarticle}
\usepackage[utf8]{inputenc}
\usepackage[T1]{fontenc}
\usepackage[spanish,es-lcroman]{babel}
\usepackage{amsmath}
\usepackage{amsthm}
\usepackage{physics}
\usepackage{tikz}
\usepackage{float}
\usepackage[autostyle,spanish=mexican]{csquotes}
\usepackage[per-mode=symbol]{siunitx}
\usepackage{gensymb}
\usepackage{multicol}
\usepackage{enumitem}
\usepackage[left=2.00cm, right=2.00cm, top=2.00cm, 
     bottom=2.00cm]{geometry}
\usepackage{Estilos/ColoresLatex}
\usepackage{makecell}

\newcommand{\textocolor}[2]{\textbf{\textcolor{#1}{#2}}}
\sisetup{per-mode=symbol}
\DeclareSIUnit[number-unit-product = {\,}]\cal{cal}
\DeclareSIUnit{\dB}{dB}
%\renewcommand{\questionlabel}{\thequestion)}
\decimalpoint
\sisetup{bracket-numbers = false}

\title{\vspace*{-2cm} Actividad 2 - Ondas sonoras\vspace{-5ex}}
\date{}

\begin{document}
\maketitle

Esta actividad otorgará hasta \textbf{10 puntos} de Evaluación Continua.
\vspace*{0.5cm}

\textbf{Instrucciones: }
\begin{itemize}
\item Anota tu nombre en cada hoja que ocupes para resolver los ejercicios.
\item Identifica el ejercicio que resuelves.
\item Resuelve detalladamente cada ejercicio, en caso de que no se tenga el desarrollo, no se tomará en cuenta como ejercicio completo, revisa con cuidado el manejo de las unidades.
\item En caso de plagios, se cancelarán todos los trabajos involucrados.
\end{itemize}


\section*{Ejercicios a resolver.}

% Tippens Sec. 25.5 Ondas sonoras audibles.
\begin{enumerate}
\item ¿Cuál es el nivel de intensidad en decibeles de un sonido que tiene una intensidad de \SI{4d-5}{\watt\per\square\meter}?
\item La intensidad de un sonido es \SI{6d-8}{\watt\per\square\meter} ¿Cuál es el nivel de intensidad?
\item A cierta distancia de un silbato se mide un sonido de \SI{60}{\dB}. ¿Cuál es la intensidad de ese sonido en \unit{\watt\per\square\meter}?
\item ¿Cuál es la intensidad de un sonido de \SI{250}{\dB}?
\item Calcula las intensidades correspondientes a sonidos de \num{10}, \num{20} y \SI{30}{\dB}.
\item Calcula los niveles de intensidad que corresponden a sonidos de \SI{1d-6}{\watt\per\square\meter}, \SI{25d-6}{\watt\per\square\meter} y \SI{300d-6}{\watt\per\square\meter}.
\end{enumerate}
Para los siguientes problemas, considera que la rapidez del sonido es \SI{343}{\meter\per\second}.
\begin{enumerate}[resume*]
\item Una fuente estacionaria de sonido emite una señal cuya frecuencia es de \SI{290}{\hertz}. ¿Cuáles son las frecuencias que oye un observador (a) que se aproxima a la fuente a \SI{20}{\meter\per\second} y (b) que se aleja de la fuente a \SI{20}{\meter\per\second}?
\item Un automóvil hace sonar una bocina a \SI{560}{\hertz} mientras se desplaza con una rapidez de \SI{15}{\meter\per\second}, primero aproximándose a un oyente estacionario y después alejándose de él con la misma rapidez. ¿Cuáles son las frecuencias que escucha el oyente?
\item En un automóvil estacionado, una persona hace sonar una bocina a \SI{400}{\hertz}. ¿Qué frecuencias escucha el conductor de un vehículo que pasa junto al primero con una rapidez de \SI{60}{\kilo\meter\per\hour}?
\item Un tren que avanza a \SI{20}{\meter\per\second} hace sonar un silbato a \SI{300}{\hertz} al pasar junto a un observador estacionario. ¿Cuáles son las frecuencias que oye el observador al pasar el tren?
\end{enumerate}
\end{document}