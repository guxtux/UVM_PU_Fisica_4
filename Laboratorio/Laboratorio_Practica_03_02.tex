\documentclass[14pt]{extarticle}
\usepackage[utf8]{inputenc}
\usepackage[T1]{fontenc}
\usepackage[spanish,es-lcroman]{babel}
\usepackage{amsmath}
\usepackage{amsthm}
\usepackage{physics}
\usepackage{tikz}
\usepackage{float}
\usepackage[autostyle,spanish=mexican]{csquotes}
\usepackage[per-mode=symbol]{siunitx}
\usepackage{gensymb}
\usepackage{multicol}
\usepackage{enumitem}
\usepackage[left=2.00cm, right=2.00cm, top=2.00cm, 
     bottom=2.00cm]{geometry}

%\renewcommand{\questionlabel}{\thequestion)}
\decimalpoint
\sisetup{bracket-numbers = false}

\title{\vspace*{-2cm} Práctica 3 - Campo visual \\  Física IV (Área II) \vspace{-5ex}}
\date{}

\begin{document}
\maketitle

\section{Resultados.}

En este apartado del reporte de la Práctica 3, deberás de incluir ya sea las fotos o esquemas del campo visual del par de estudiantes, por cada ojo.

\section{Análisis de resultados.}

\begin{enumerate}
\item Realiza lo siguiente:
\begin{enumerate}
\itemsep0.5em 
\item Coloca cada campo visual del ojo derecho y compara el mismo del par de estudiantes.
\item ¿Qué diferencias hay?
\item ¿Las regiones donde se localizan los escotomas (puntos ciegos) son similares?
\item ¿Algún integrante utiliza lentes?
\end{enumerate}
\item Repite lo mismo para el ojo izquierdo y responde las mismas preguntas.
\item Ahora coloca los dos campos visuales. ¿Qué similitud hay? Con elementos de la física ¿Cómo explicarías las diferencias en los campos visuales del par de integrantes?
\end{enumerate}

\section{Conclusiones.}

Responde con argumentos y los resultados obtenidos si se logró el objetivo de la Práctica, responde además si la hipótesis planteada al inicio es verdadera o no.

Considera que te puedes apoyar con la investigación preliminar sobre el ojo, el campo visual periférico y la visión binocular.

\section{Formato para el Reporte.}

El reporte es INDIVIDUAL, deberás de ocupar el formato en Word para incluir cada apartado y enviarlo por Teams antes de la fecha de cierre: \textbf{Lunes 27 de noviembre a las 8 pm}.
\end{document}