\documentclass[14pt]{beamer}
\usepackage{./Estilos/BeamerUVM}
\usepackage{./Estilos/ColoresLatex}
\usetheme{Madrid}
\usecolortheme{default}
%\useoutertheme{default}
\setbeamercovered{invisible}
% or whatever (possibly just delete it)
\setbeamertemplate{section in toc}[sections numbered]
\setbeamertemplate{subsection in toc}[subsections numbered]
\setbeamertemplate{subsection in toc}{\leavevmode\leftskip=3.2em\rlap{\hskip-2em\inserttocsectionnumber.\inserttocsubsectionnumber}\inserttocsubsection\par}
% \setbeamercolor{section in toc}{fg=blue}
% \setbeamercolor{subsection in toc}{fg=blue}
% \setbeamercolor{frametitle}{fg=blue}
\setbeamertemplate{caption}[numbered]

\setbeamertemplate{footline}
\beamertemplatenavigationsymbolsempty
\setbeamertemplate{headline}{}


\makeatletter
% \setbeamercolor{section in foot}{bg=gray!30, fg=black!90!orange}
% \setbeamercolor{subsection in foot}{bg=blue!30}
% \setbeamercolor{date in foot}{bg=black}
\setbeamertemplate{footline}
{
  \leavevmode%
  \hbox{%
  \begin{beamercolorbox}[wd=.333333\paperwidth,ht=2.25ex,dp=1ex,center]{section in foot}%
    \usebeamerfont{section in foot} {\insertsection}
  \end{beamercolorbox}%
  \begin{beamercolorbox}[wd=.333333\paperwidth,ht=2.25ex,dp=1ex,center]{subsection in foot}%
    \usebeamerfont{subsection in foot}  \insertsubsection
  \end{beamercolorbox}%
  \begin{beamercolorbox}[wd=.333333\paperwidth,ht=2.25ex,dp=1ex,right]{date in head/foot}%
    \usebeamerfont{date in head/foot} \insertshortdate{} \hspace*{2em}
    \insertframenumber{} / \inserttotalframenumber \hspace*{2ex} 
  \end{beamercolorbox}}%
  \vskip0pt%
}
\makeatother

\makeatletter
\patchcmd{\beamer@sectionintoc}{\vskip1.5em}{\vskip0.8em}{}{}
\makeatother

% \usefonttheme{serif}
\usepackage[clock]{ifsym}

\sisetup{per-mode=symbol}
\resetcounteronoverlays{saveenumi}

\title{\Large{Presentación syllabus} \\ \normalsize{Física IV (área II)}}
\date{31 de agosto de 2023}

\begin{document}
\maketitle

\section*{Contenido}
\frame[allowframebreaks]{\frametitle{Contenido} \tableofcontents[currentsection, hideallsubsections]}

\section{Objetivos}
\frame{\tableofcontents[currentsection, hideothersubsections]}
\subsection{Metas esperadas}

\begin{frame}
\frametitle{Objetivos del curso}
El alumno aplicará los conceptos, principios, leyes, lenguajes de representación y metodologías de la Física a partir de la comprensión y explicación de fenómenos físicos inherentes en procesos químicos y biológicos específicos, 
\end{frame}
\begin{frame}
\frametitle{Objetivos del curso}
Con el fin de que emplee los instrumentos tecnológicos de punta de manera razonada (inductivo, deductivo y abductivo) y argumentada científicamente, así como con una actitud responsable y propositiva.
\end{frame}

\section{Evaluación}
\frame{\tableofcontents[currentsection, hideothersubsections]}
\subsection{Examen parcial}

\begin{frame}
\frametitle{Examen parcial}
Se aplicarán cuatro exámenes parciales durante el ciclo escolar.
\\
\bigskip
Las fechas ya están definidas en el calendario UVM.
\end{frame}
\begin{frame}
\frametitle{Fechas de los exámenes}
\begin{table}
\renewcommand{\arraystretch}{1.1}
\centering
\begin{tabular}{c | c | c}
Examen & Fecha & Unidad \\ \hline
$1$ & 16 al 27 octubre 2023 & $1$ \\ \hline
$2$ & 4 al 15 diciembre 2023 & $1$ \\ \hline
$3$ & 26 de febrero al 8 de marzo 2024 & $2$ \\ \hline
$4$ & 29 de abril al 10 de mayo 2024 & $2$ \\ \hline
\end{tabular}
\end{table}
\end{frame}
\begin{frame}
\frametitle{Peso del examen}
El examen representa el $60\%$ de la calificación del parcial.
\end{frame}

\subsection{Evaluación continua}

\begin{frame}
\frametitle{Evaluación continua}
La evaluación continua se considera el conjunto de actividades como: 
\setbeamercolor{item projected}{bg=aqua,fg=black}
\setbeamertemplate{enumerate items}{%
\usebeamercolor[bg]{item projected}%
\raisebox{1.5pt}{\colorbox{bg}{\color{fg}\footnotesize\insertenumlabel}}%
}
\begin{enumerate}[<+->]
\item Trabajos de investigación.
\item Ejercicios semanales.
\item Ejercicios en clase.
\item Guías de estudio.
\end{enumerate}
\end{frame}
\begin{frame}
\frametitle{Trabajo de investigación}
Serán actividades en donde se espera que luego de una consulta de varias fuentes de información, el alumno prepare ya sea: una línea de tiempo, una revisión biográfica y de aportaciones de científicos destacados en un área, etc.
\end{frame}
\begin{frame}
\frametitle{Preparación del trabajo}
Se ocupará una rúbrica de evaluación en donde se indicarán los componentes de cada actividad, así como el nivel de desempeño esperado.
\end{frame}
\begin{frame}
\frametitle{Ejercicios semanales}
Con la finalidad de repasar cada tema, se dejará una lista de ejercicios cada semana.
\\
\bigskip
\pause
El alumno resolverá cada ejercicio de la manera más detallada posible.
\end{frame}
\begin{frame}
\frametitle{Entrega de los ejercicios semanales}
Mediante una asignación en Teams se hará el envío de la solución de los ejercicios, \pause estableciendo una fecha de entrega.
\end{frame}
\begin{frame}
\frametitle{Entrega de los ejercicios semanales}
Contarán con el suficiente tiempo para resolver los ejercicios, pudiendo hacer consultas, preguntas, y con ello, se entregue la mayor cantidad de ejercicios.
\end{frame}
\begin{frame}
\frametitle{Evaluación de los ejercicios semanales}
Un ejercicio aportará un punto siempre y cuando esté bien resuelto.
\\
\bigskip
\pause
En caso de que se tenga un avance en la solución, pero el resultado no sea el esperado, se otorgará una parte del punto.
\end{frame}
\begin{frame}
\frametitle{Retroalimentación}
Se devolverán los ejercicios con comentarios y observaciones, esperando que en los siguientes se atiendan y logren la mayor calificación.
\end{frame}
\begin{frame}
\frametitle{Solución personal}
Se espera que el trabajo en cada solución sea personal.
\\
\bigskip
\pause
En caso de identificar copias en la solución, se cancelará el total de copias involucradas.
\end{frame}
\begin{frame}
\frametitle{Entregas extemporáneas}
Como se establecerá una fecha con el suficiente tiempo para resolver los ejercicos, no se recibirán trabajos extemporáneos.
\end{frame}
\begin{frame}
\frametitle{¿Qué hago si no entregue mis ejercicios?}
En caso de no haber entregado una actividad, el alumno deberá de cubrir la llamada \textocolor{cobalt}{actividad de recuperación}.
\end{frame}
\begin{frame}
\frametitle{Actividad de recuperación}
La actividad de recuperación consistirá en una serie de ejercicios similares que deberán de resolverse y enviarse en la asignación que abra el Profesor.
\end{frame}
\begin{frame}
\frametitle{Ejercicios en clase}
Durante una sesión de clase, el Profesor podrá dejar un ejercicio para que el grupo lo complete y prepare la solución en su cuaderno.
\end{frame}
\begin{frame}
\frametitle{Ejercicios en clase}
El Profesor firmará la actividad como evidencia de trabajo, \pause el alumno deberá de enviar una foto de la hoja firmada a la asignación en Teams.
\end{frame}
\begin{frame}
\frametitle{Guías de estudio}
Previo a las fechas del examen, el alumno elaborará una guía de estudio con los temas revisados durante las sesiones.
\end{frame}
\begin{frame}
\frametitle{Peso de la evaluación continua}
Se sumarán las actividades realizadas y enviadas, el peso que se considera para la Evaluación Continua es del $30\%$ de la calificación del examen parcial.
\end{frame}
\begin{frame}
\frametitle{Precisión importante}
La asignatura de Física IV (Área II) se considera como una asignatura de tipo teórico - práctico, \pause por Reglamento el peso del Examen y de la Evaluación Continua es del $70\%$ de la calificación del parcial.
\end{frame}

\subsection{Laboratorio}

\begin{frame}
\frametitle{Trabajo en el Laboratorio}
El trabajo en Laboratorio consistirá en el montaje de prácticas dirigidas en las sesiones de clase.
\end{frame}
% \begin{frame}
% \frametitle{Etapas de trabajo}
% \setbeamercolor{item projected}{bg=aquamarine,fg=black}
% \setbeamertemplate{enumerate items}{%
% \usebeamercolor[bg]{item projected}%
% \raisebox{1.5pt}{\colorbox{bg}{\color{fg}\footnotesize\insertenumlabel}}%
% }
% \begin{enumerate}[<+->]
% \item Discusión de la práctica.
% \item Montaje experimental.
% \item Interpretación de resultados y reporte.
% \end{enumerate}
% \end{frame}
% \begin{frame}
% \frametitle{Discusión de la práctica}
% Durante el año escolar se realizarán varias prácticas, en una primera sesión se discutirá sobre el objetivo de la misma, así como el marco teórico para su comprensión.
% \end{frame}
% \begin{frame}
% \frametitle{Discusión de la práctica}
% El alumno deberá de complementar la revisión durante la semana, de tal manera que en la clase de montaje, tendrá el soporte de conocimiento necesario para realizar la práctica.
% \end{frame}
% \begin{frame}
% \frametitle{Montaje de la práctica}
% En una segunda sesión se realizará el montaje de la práctica durante la clase, esto implica trabajo en equipo.
% \end{frame}
% \begin{frame}
% \frametitle{Recabando datos e información}
% Cada equipo de trabajo, deberá de recolectar los datos de la práctica, de tal manera que con trabajo adicional durante la semana, llegará a la siguiente sesión con un trabajo preliminar.
% \end{frame}
% \begin{frame}
% \frametitle{Interpretación de datos y reporte}
% En la tercera sesión, cada equipo de trabajo revisará los resultados preliminares, discutirán sobre los hechos hallados y revisarán su congruencia con el marco téorico.
% \end{frame}
% \begin{frame}
% \frametitle{Elaboración del reporte}
% Una vez revisada la parte de interpretación, cada alumno realizará un reporte de la práctica.
% \\
% \bigskip
% \pause
% Se dispondrá de una rúbrica para la evaluación del reporte de la práctica.
% \end{frame}
\begin{frame}
\frametitle{Calificación de Laboratorio}
El peso de la calificación de Laboratorio corresponde al $30 \%$ de la calificación de cada examen parcial.
% \\
% \bigskip
% \pause
% Se tendrán cuatro exámenes parciales durante el ciclo escolar.
\end{frame}

\section{Temas del curso}
\frame[allowframebreaks]{\tableofcontents[currentsection, hideothersubsections]}
\subsection{Física de la visión y audición}

\begin{frame}
\frametitle{Contenido de la unidad 1}
\setbeamercolor{item projected}{bg=alizarin,fg=aliceblue}
\setbeamertemplate{enumerate items}{%
\usebeamercolor[bg]{item projected}%
\raisebox{1.5pt}{\colorbox{bg}{\color{fg}\footnotesize\insertenumlabel}}%
}
\begin{enumerate}[<+->]
\item Ondas. Características: periodo, frecuencia, velocidad, amplitud, intensidad, entre otros.
\item Fenómenos sonoros: reflexión, difracción, resonancia, superposición de ondas, entre otros.
\seti
\end{enumerate}
\end{frame}
\begin{frame}
\frametitle{Contenido de la unidad 1}
\setbeamercolor{item projected}{bg=alizarin,fg=aliceblue}
\setbeamertemplate{enumerate items}{%
\usebeamercolor[bg]{item projected}%
\raisebox{1.5pt}{\colorbox{bg}{\color{fg}\footnotesize\insertenumlabel}}%
}
\begin{enumerate}[<+->]
\conti
\item Oído y transferencia de energía.
\item Efecto Doppler.
\item Ondas electromagnéticas (luz visible)
\item Principio de Huygens.
\seti
\end{enumerate}
\end{frame}
\begin{frame}
\frametitle{Contenido de la unidad 1}
\setbeamercolor{item projected}{bg=alizarin,fg=aliceblue}
\setbeamertemplate{enumerate items}{%
\usebeamercolor[bg]{item projected}%
\raisebox{1.5pt}{\colorbox{bg}{\color{fg}\footnotesize\insertenumlabel}}%
}
\begin{enumerate}[<+->]
\conti
\item Óptica geométrica.
\item Refracción (índice de refracción, ley de Snell).
\item Lentes delgadas.
\item Formación de imágenes en lentes.
\seti
\end{enumerate}
\end{frame}
\begin{frame}
\frametitle{Contenido de la unidad 1}
\setbeamercolor{item projected}{bg=alizarin,fg=aliceblue}
\setbeamertemplate{enumerate items}{%
\usebeamercolor[bg]{item projected}%
\raisebox{1.5pt}{\colorbox{bg}{\color{fg}\footnotesize\insertenumlabel}}%
}
\begin{enumerate}[<+->]
\conti
\item Miopía.
\item Hipermetropía.
\item Astigmatismo.
\item Instrumentación biomédica: Estetoscopio, endoscopio, microscopio, aparato para realizar ultrasonido, entre otros.
\end{enumerate}
\end{frame}

\subsection{Fluidos y pulsos eléctricos}
\begin{frame}
\frametitle{Contenido de la unidad 2}
\setbeamercolor{item projected}{bg=bananayellow,fg=ao}
\setbeamertemplate{enumerate items}{%
\usebeamercolor[bg]{item projected}%
\raisebox{1.5pt}{\colorbox{bg}{\color{fg}\footnotesize\insertenumlabel}}%
}
\begin{enumerate}[<+->]
\item Ecuación de continuidad.
\item Ecuación de Bernoulli.
\item Ecuación de Poiseuille.
\seti
\end{enumerate}
\end{frame}
\begin{frame}
\frametitle{Contenido de la unidad 2}
\setbeamercolor{item projected}{bg=bananayellow,fg=ao}
\setbeamertemplate{enumerate items}{%
\usebeamercolor[bg]{item projected}%
\raisebox{1.5pt}{\colorbox{bg}{\color{fg}\footnotesize\insertenumlabel}}%
}
\begin{enumerate}[<+->]
\item Corriente directa y alterna.
\item Ley de Ohm.
\item Circuitos eléctricos. Mixtos, RC y RCL.
\item Impedancia eléctrica.
\item Potencial de acción.
\seti
\end{enumerate}
\end{frame}
\begin{frame}
\frametitle{Contenido de la unidad 2}
\setbeamercolor{item projected}{bg=bananayellow,fg=ao}
\setbeamertemplate{enumerate items}{%
\usebeamercolor[bg]{item projected}%
\raisebox{1.5pt}{\colorbox{bg}{\color{fg}\footnotesize\insertenumlabel}}%
}
\begin{enumerate}[<+->]
\item Seguridad eléctrica. La importancia de la conexión a tierra física.
\item Instrumentación biomédica: Esfigmomanómetro, electrocardiógrafo, desfibrilador, encefalógrafo, marcapasos, entre otros.
\end{enumerate}
\end{frame}

\section{Consideraciones importantes}
\frame[allowframebreaks]{\tableofcontents[currentsection, hideothersubsections]}
\subsection{Criterios de exención}

\begin{frame}
\frametitle{Criterio de exención}
Para que el alumno pueda quedar \textocolor{ao}{exento de presentar el examen final ordinario} de la asignatura, \pause el promedio mínimo de la calificación que obtenga, durante el ciclo escolar, deberá ser de 8 (ocho) y tener, al menos, el $80\%$ de asistencia.
\end{frame}

\subsection{De las calificaciones}

\begin{frame}
\frametitle{Asignación de las calificaciones}
Al promediar las calificaciones de los semestres, se obtendrá la calificación final:
\setbeamercolor{item projected}{bg=armygreen,fg=white}
\setbeamertemplate{enumerate items}{%
\usebeamercolor[bg]{item projected}%
\raisebox{1.5pt}{\colorbox{bg}{\color{fg}\footnotesize\insertenumlabel}}%
}
\begin{enumerate}[<+->]
\item Si el \textocolor{awesome}{promedio es igual o mayor} a $8.0$ podrá exentar el examen final de acuerdo con el criterio de exención.
\seti
\end{enumerate}
\end{frame}
\begin{frame}
\frametitle{Asignación de las calificaciones}
\setbeamercolor{item projected}{bg=armygreen,fg=white}
\setbeamertemplate{enumerate items}{%
\usebeamercolor[bg]{item projected}%
\raisebox{1.5pt}{\colorbox{bg}{\color{fg}\footnotesize\insertenumlabel}}%
}
\begin{enumerate}[<+->]
\conti
\item Si el \textocolor{bazaar}{promedio final es menor} de 8.0, el estudiante deberá presentar un examen final (primera y/o segunda vuelta).
\end{enumerate}
\end{frame}
\begin{frame}
\frametitle{Importancia de la asistencia}
Podrán presentar examen final ordinario de primera y/o segundas vueltas, previa identificación, los alumnos que reúnan el $80\%$ de asistencias y que no hayan quedado exentos.
\end{frame}
\begin{frame}
\frametitle{Asignación de las calificaciones}    
En caso de que la calificación del examen final, ya sea en primera o segunda vuelta, promediada con la calificación anual del curso, \textocolor{burgundy}{no arroje un promedio aprobatorio}, \pause el estudiante \textocolor{coquelicot}{deberá presentar examen extraordinario.}
\end{frame}
\begin{frame}
\frametitle{Asignación de las calificaciones}
Para tener derecho a las evaluaciones parciales y exámenes ordinarios de fin de curso, se deberá cumplir, necesariamente, el requisito de acumular como mínimo el $80\%$ de asistencia a clases.
\end{frame}

% \begin{frame}
% \frametitle{Prácticas para la segunda parte}
% Como tema para la segunda parte del curso, se revisarán actividades de calor, temperatura, electricidad y energía.
% \end{frame}
% \begin{frame}
% \frametitle{Prácticas para la segunda parte}
% \setbeamercolor{item projected}{bg=darkscarlet,fg=white}
% \setbeamertemplate{enumerate items}{%
% \usebeamercolor[bg]{item projected}%
% \raisebox{1.5pt}{\colorbox{bg}{\color{fg}\footnotesize\insertenumlabel}}%
% }
% \begin{enumerate}[<+->]
% \item Ley de Faraday.
% \item Calor específico.
% \item Transformación de energía.
% \end{enumerate}
% \end{frame}

% \section{Trabajo en Laboratorio}
% \frame{\tableofcontents[currentsection, hideothersubsections]}

\section{Asistencia a clases}
\frame{\tableofcontents[currentsection, hideothersubsections]}
\subsection{Pase de asistencia}

\begin{frame}
\frametitle{Pase de asistencia}
Se realizará el \textocolor{red}{pase de asistencia} luego de la tolerancia para ingresar a la clase, que es de 5 minutos.
\end{frame}
\begin{frame}
\frametitle{Pase de asistencia}
En caso de que la(el) alumna(o) no se encuentre en la clase al momento del pase de asistencia, se le marcará como inasistencia, así haya ingresado luego de completar el pase.
\end{frame}
\begin{frame}
\frametitle{¿Qué debo hacer si falto a clase?}
El alumno deberá de regularizarse lo más pronto posible, atendiendo las notas de clase, revisando los ejercicios a resolver, etc.
\end{frame}

\section{Lineamientos para la clase}
\frame[allowframebreaks]{\tableofcontents[currentsection, hideothersubsections]}
\subsection{Importante de atender}

\begin{frame}
\frametitle{En cada clase}
\setbeamercolor{item projected}{bg=blue-violet,fg=white}
\setbeamertemplate{enumerate items}{%
\usebeamercolor[bg]{item projected}%
\raisebox{1.5pt}{\colorbox{bg}{\color{fg}\footnotesize\insertenumlabel}}%
}
\begin{enumerate}[<+->]
\item No se permite el uso de teléfonos celulares dentro del aula, en caso de atender una llamada, avisa al Profesor para que salgas al pasillo a responder.
\seti
\end{enumerate}
\end{frame}
\begin{frame}
\frametitle{En cada clase}
\setbeamercolor{item projected}{bg=blue-violet,fg=white}
\setbeamertemplate{enumerate items}{%
\usebeamercolor[bg]{item projected}%
\raisebox{1.5pt}{\colorbox{bg}{\color{fg}\footnotesize\insertenumlabel}}%
}
\begin{enumerate}[<+->]
\conti
\item No se permite el uso de audífonos. Para el desarrollo de la clase se requiere que estés atenta(o) en todo momento.
\seti
\end{enumerate}
\end{frame}
\begin{frame}
\frametitle{En cada clase}
\setbeamercolor{item projected}{bg=blue-violet,fg=white}
\setbeamertemplate{enumerate items}{%
\usebeamercolor[bg]{item projected}%
\raisebox{1.5pt}{\colorbox{bg}{\color{fg}\footnotesize\insertenumlabel}}%
}
\begin{enumerate}[<+->]
\conti
\item La clase es un escenario dinámico de interacción en el proceso de enseñanza – aprendizaje, en caso de que tengas una duda o comentario, levanta la mano para que el Profesor otorgue la palabra de manera ordenada.
\seti
\end{enumerate}
\end{frame}
\begin{frame}
\frametitle{En cada clase}
\setbeamercolor{item projected}{bg=blue-violet,fg=white}
\setbeamertemplate{enumerate items}{%
\usebeamercolor[bg]{item projected}%
\raisebox{1.5pt}{\colorbox{bg}{\color{fg}\footnotesize\insertenumlabel}}%
}
\begin{enumerate}[<+->]
\conti
\item En el caso de una inasistencia, la(el) alumna(o) deberá de regularizarse lo más pronto posible.
\seti
\end{enumerate}
\end{frame}
\begin{frame}
\frametitle{En cada clase}
\setbeamercolor{item projected}{bg=blue-violet,fg=white}
\setbeamertemplate{enumerate items}{%
\usebeamercolor[bg]{item projected}%
\raisebox{1.5pt}{\colorbox{bg}{\color{fg}\footnotesize\insertenumlabel}}%
}
\begin{enumerate}[<+->]
\conti
\item Para cada actividad de Evaluación continua se abrirá una asignación en Teams, con un día y hora establecido para el envío, la plataforma no permite el envío extemporáneo de alguna actividad.
\seti
\end{enumerate}
\end{frame}
\begin{frame}
\frametitle{En cada clase}
\setbeamercolor{item projected}{bg=blue-violet,fg=white}
\setbeamertemplate{enumerate items}{%
\usebeamercolor[bg]{item projected}%
\raisebox{1.5pt}{\colorbox{bg}{\color{fg}\footnotesize\insertenumlabel}}%
}
\begin{enumerate}[<+->]
\conti
\item Cuando el Profesor lo indique, el(la) alumno(a) enviará por mensaje directo alguna actividad de manera extemporánea, que se calificará sobre 7 (siete).
\end{enumerate}
\end{frame}

% \begin{frame}
% \frametitle{De la asistencia}
% La sesión de Laboratorio se lleva a cabo una vez a la semana.
% \\
% \bigskip
% \pause
% Siendo indispensable asistir a cada una de las clases de Laboratorio.
% \end{frame}
    
% \subsection{Elementos necesarios}

% \begin{frame}
% \frametitle{Elementos necesarios}
% Para el trabajo en laboratorio, es necesario utilizar una BATA BLANCA.
% \\
% \bigskip
% \pause
% Revisaremos que en el Manual de Seguridad y Normas de Higiene el Laboratorio, el uso de Bata es requisito.
% \end{frame}
% \begin{frame}
% \frametitle{¿Qué pasa si no traigo la bata?}
% En caso de que no se tenga la bata, la(el) alumna(o) no podrá permanecer en el Laboratorio, \pause se notificará a la Coordinación Académica para que los acompañe a la Biblioteca en donde tendrán que realizar una actividad entregable al concluir la clase.
% \end{frame}
% \begin{frame}
% \frametitle{¿Qué pasa si no traigo la bata?}
% Deberán de completar la actividad, pero no podrán reponer la actividad de Laboratorio y contarán con inasistencia.
% \\
% \bigskip
% \pause
% Es recomendable que se prevengan para contar con una bata blanca.
% \end{frame}

% \begin{frame}
% \frametitle{El Reglamento}
% Revisaremos en la siguiente sesión el Manual de Seguridad y las Normas de Higiene para el Laboratorio.
% \end{frame}

\subsection{Primera actividad}

\begin{frame}
\frametitle{Primera actividad}
Será necesario que cada alumno incluya en su cuardeno, carpeta de notas, etc. la \textocolor{ao}{síntesis de la asignatura}, con la firma del alumno así como del Tutor.
\end{frame}
\begin{frame}
\frametitle{Primera actividad}
De esta manera, se tendrá la claridad en el trabajo, la didáctica en la clase, el reglamento y las actividades a realizar durante el año escolar.
\end{frame}
\begin{frame}
\frametitle{Actividad que cuenta}
El presentar la síntesis de la asignatura debidamente firmada, contará en el puntaje de evaluación para el Laboratorio.
\end{frame}
\begin{frame}
\frametitle{Entrega de la actividad}
Se abrirá una asignación en Teams para el envío de una foto de la síntesis firmada por el alumno y el Tutor, además de tenerla en su cuaderno de trabajo.
\end{frame}
% \begin{frame}
% \frametitle{Primeras actividades}
% Se informa al grupo que se está acabando de afinar el espacio de Laboratorio para las clases, \pause por lo que las primeras tres actividades se llevarán a cabo en el aula.
% \end{frame}
% \begin{frame}
% \frametitle{Primeras actividades}
% Estas primeras actividades contabilizarán como prácticas y por lo tanto, sumarán puntos para el primer examen parcial.
% \end{frame}
% \begin{frame}
% \frametitle{Consideración importante}
% Como ya se mencionó, se tendrá una clase de Laboratorio a la semana.
% \\
% \bigskip
% \pause
% No asistir a una clase compromete la evaluación, ya que no se podrá reponer la actividad a la que se haya asistido.
% \end{frame}


\end{document}