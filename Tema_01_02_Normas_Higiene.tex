\documentclass[14pt]{beamer}
\usepackage{./Estilos/BeamerUVM}
\usepackage{./Estilos/ColoresLatex}
\usetheme{Madrid}
\usecolortheme{default}
%\useoutertheme{default}
\setbeamercovered{invisible}
% or whatever (possibly just delete it)
\setbeamertemplate{section in toc}[sections numbered]
\setbeamertemplate{subsection in toc}[subsections numbered]
\setbeamertemplate{subsection in toc}{\leavevmode\leftskip=3.2em\rlap{\hskip-2em\inserttocsectionnumber.\inserttocsubsectionnumber}\inserttocsubsection\par}
% \setbeamercolor{section in toc}{fg=blue}
% \setbeamercolor{subsection in toc}{fg=blue}
% \setbeamercolor{frametitle}{fg=blue}
\setbeamertemplate{caption}[numbered]

\setbeamertemplate{footline}
\beamertemplatenavigationsymbolsempty
\setbeamertemplate{headline}{}


\makeatletter
% \setbeamercolor{section in foot}{bg=gray!30, fg=black!90!orange}
% \setbeamercolor{subsection in foot}{bg=blue!30}
% \setbeamercolor{date in foot}{bg=black}
\setbeamertemplate{footline}
{
  \leavevmode%
  \hbox{%
  \begin{beamercolorbox}[wd=.333333\paperwidth,ht=2.25ex,dp=1ex,center]{section in foot}%
    \usebeamerfont{section in foot} {\insertsection}
  \end{beamercolorbox}%
  \begin{beamercolorbox}[wd=.333333\paperwidth,ht=2.25ex,dp=1ex,center]{subsection in foot}%
    \usebeamerfont{subsection in foot}  \insertsubsection
  \end{beamercolorbox}%
  \begin{beamercolorbox}[wd=.333333\paperwidth,ht=2.25ex,dp=1ex,right]{date in head/foot}%
    \usebeamerfont{date in head/foot} \insertshortdate{} \hspace*{2em}
    \insertframenumber{} / \inserttotalframenumber \hspace*{2ex} 
  \end{beamercolorbox}}%
  \vskip0pt%
}
\makeatother

\makeatletter
\patchcmd{\beamer@sectionintoc}{\vskip1.5em}{\vskip0.8em}{}{}
\makeatother

% \usefonttheme{serif}
\usepackage[clock]{ifsym}

\sisetup{per-mode=symbol}
\resetcounteronoverlays{saveenumi}

\title{\Large{Normas de Higiene} \\ \normalsize{Física III}}
\date{6 de septiembre de 2023}

\renewcommand\cellset{\renewcommand\arraystretch{0.7}%
\setlength\extrarowheight{0pt}}

\begin{document}
\maketitle

\section*{Contenido}
\frame{\frametitle{Contenido} \tableofcontents[currentsection, hideallsubsections]}

\section{Normas de Higiene}
\frame{\tableofcontents[currentsection, hideothersubsections]}
\subsection{Elementos a considerar}

\begin{frame}
\frametitle{Primer punto}
Es importante que tomes en cuenta estas recomendaciones ya que así evitaremos muchos accidentes
\end{frame}
\begin{frame}
\frametitle{Norma 1}
Toda persona que ingrese a los laboratorios lo deberá hacer con una bata blanca de manga larga (100\% algodón) y abotonada.
\end{frame}
\begin{frame}
\frametitle{Norma 2}
El uso de guantes y lentes de seguridad con protección lateral dependerá de lo que indique el Profesor.
\end{frame}
\begin{frame}
\frametitle{Norma 3}
Queda estrictamente PROHIBIDO FUMAR y/o COMER y /o BEBER durante la estancia en el laboratorio, \pause no olvides que al iniciar la práctica los alimentos y bebidas se consideran como reactivos.
\end{frame}
\begin{frame}
\frametitle{Norma 4}
No se podrá iniciar y/o realizar ninguna práctica si el Profesor no está presente.
\end{frame}
\begin{frame}
\frametitle{Norma 5}
Mantenga el área de trabajo limpia y libre de objetos innecesarios, sea ordenado.
\end{frame}
\begin{frame}
\frametitle{Norma 6}
Los materiales y equipos sea cual fuere su naturaleza deberán ser usados con extrema precaución apegándose a los respectivos manuales de operación.
\end{frame}
\begin{frame}
\frametitle{Norma 7}
El número máximo de estudiantes que podrán intervenir en las prácticas durante una misma sesión será de 25 personas.
\end{frame}
\begin{frame}
\frametitle{Norma 8}
Informar al Profesor INMEDIATAMENTE en caso de algún accidente.
\end{frame}
\begin{frame}
\frametitle{Norma 9}
Evitar el uso de lentes de contacto cuando se manipulen sustancias químicas.
\end{frame}
\begin{frame}
\frametitle{Norma 10}
Utiliza zapatos cerrados, evita usar sandalias.
\end{frame}
\begin{frame}
\frametitle{Norma 11}
No tirar los desechos de las sustancias por el desagüe, haz uso los contenedores de plástico que se encuentran en los laboratorios.
\end{frame}
\begin{frame}
\frametitle{Norma 12}
Antes de encender cualquier flama se deberá verificar que no existan vapores de disolventes ni fugas de gas.
\\
\bigskip
\pause
El Profesor deberá de supervisar y autorizar el encendido de fuego.
\end{frame}
\begin{frame}
\frametitle{Norma 13}
Nunca dejar encendido un mechero y/o lámpara de alcohol innecesariamente.
\end{frame}
\begin{frame}
\frametitle{Norma 14}
Atarse el cabello largo para evitar accidentes con la llama del mechero.
\end{frame}
\begin{frame}
\frametitle{Norma 15}
Hacer uso de campana de extracción, regadera de seguridad y extractor, cuando sea necesario (laboratorio de química)
\end{frame}
\begin{frame}
\frametitle{Norma 16}
Todos los laboratorios cuentan con extintores y botiquín en caso de alguna emergencia.
\end{frame}
\begin{frame}
\frametitle{Norma 17}
Una vez que se utilizó la toma de gas, el profesor deberá verificar que la llave de seguridad se encuentre cerrada.
\end{frame}

\section{Reglamento Laboratorio}
\frame{\tableofcontents[currentsection, hideothersubsections]}
\subsection{Capítulo 1}

\begin{frame}
\frametitle{De su aplicación y definición}
\textocolor{ao}{Artículo 1.} El presente Reglamento se aplica a los estudiantes que hacen uso de los distintos talleres y laboratorios ubicados dentro de los campus, que sirven de apoyo en las asignaturas prácticas.
\end{frame}

\subsection{Capítulo 2}

\begin{frame}
\frametitle{De la seguridad}
\textocolor{ao}{Artículo 4.} Todos los estudiantes que asistan a una práctica de laboratorio o taller deberán observar estrictamente las normas de seguridad e higiene establecidas para su uso en los propios talleres y
laboratorios.
\end{frame}
\begin{frame}
\frametitle{De la seguridad}
\textocolor{ao}{Artículo 9.} Deberá mantenerse el área de trabajo en orden, despejada de objetos que  obstaculicen las prácticas, la libre circulación o que puedan provocar algún accidente.
\end{frame}
\begin{frame}
\frametitle{De la seguridad}
\textocolor{ao}{Artículo 10.} El docente de la asignatura deberá estar presente en el laboratorio durante todas las prácticas que solicite conforme a la materia que imparta.
\end{frame}
\begin{frame}
\frametitle{De la seguridad}
\textocolor{ao}{Artículo 12.} Está prohibido que los estudiantes ingresen al almacén de cualquier taller y laboratorio, los únicos autorizados serán el docente, el auxiliar del taller y laboratorio o el personal designado por el campus.
\end{frame}
\begin{frame}
\frametitle{De la seguridad}
\textocolor{ao}{Artículo 13.} En caso de ocurrir cualquier accidente dentro del taller y laboratorio como cortaduras, quemaduras, picaduras, salpicaduras, etc., \pause el estudiante o la primera persona que detecte esta situación deberá avisar inmediatamente al docente o al respectivo responsable en turno del taller y laboratorio para que se puedan tomar las medidas pertinentes.
\end{frame}
\begin{frame}
\frametitle{De la seguridad}
\textocolor{ao}{Artículo 14.} Toda persona que ingrese al taller y laboratorio, está obligada a adoptar las medidas preventivas necesarias para su protección, \pause así como conocer la simbología utilizada y respetar la señalización indicada.
\end{frame}

\subsection{Capítulo 3}

\begin{frame}
\frametitle{De la realización de las prácticas}
\textocolor{ao}{Artículo 25.} En ningún caso el estudiante podrá ingresar al laboratorio después de 10 minutos de iniciada la práctica.
\end{frame}
\begin{frame}
\frametitle{De la realización de las prácticas}
\textocolor{ao}{Artículo 26.} Es responsabilidad del estudiante acudir al taller y laboratorio con el material solicitado previamente por el docente de lo contrario no podrá realizar la práctica correspondiente.
\end{frame}

\subsection{Capítulo 4}

\begin{frame}
\frametitle{De las prácticas extracurriculares}
En este ciclo NO SE TENDRÁN prácticas extracurriculares, por esa razón se ha dividido cada práctica en 3 sesiones.
\end{frame}

\subsection{Capítulo 5}

\begin{frame}
\frametitle{De la disciplina}
\textocolor{ao}{Artículo 37.} El salir del taller y laboratorio llevando consigo material o equipo sin autorización ya sea en forma deliberada o accidental se considerará como falta grave, por lo que se deberá reportar de inmediato a la coordinación académica correspondiente.
\end{frame}
\begin{frame}
\frametitle{De la disciplina}
Si el alumno sale del laboratorio sin avisar al Profesor, se notificará a la Coordinación Académica del abandono de la actividad, por lo que el alumno no podrá completarla y tendrá como calificación CERO.
\end{frame}
\begin{frame}
\frametitle{De la disciplina}
\textocolor{ao}{Artículo 41.} En los talleres y laboratorios se deberá mantener el orden y una conducta apropiada.
\end{frame}

\subsection{Capítulo 6}

\begin{frame}
\frametitle{De las sanciones}
\textocolor{ao}{Artículo 42.} Cuando el usuario no haya observado las normas que rigen el presente reglamento se estará a lo dispuesto en la disposición normativa que resulte aplicable.
\end{frame}

\end{document}