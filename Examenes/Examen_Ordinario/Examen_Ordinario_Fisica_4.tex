\documentclass[12pt, letter]{exam}
\usepackage[utf8]{inputenc}
\usepackage[T1]{fontenc}
\usepackage[spanish]{babel}
\usepackage{amsmath}
\usepackage{amsthm}
\usepackage{physics}
\usepackage{tikz}
\usepackage{float}
\usepackage{siunitx}
\usepackage{multicol}
\usepackage[left=2.00cm, right=2.00cm, top=2.00cm, 
     bottom=2.00cm]{geometry}
\usepackage{pdfpages}

% \renewcommand{\questionlabel}{\thequestion}
\decimalpoint

\setlength{\belowdisplayskip}{-0.5pt}

\usepackage{tasks}
\settasks{
    label=\Alph*), 
    label-align=left,
    item-indent={20pt}, 
    column-sep={4pt},
    label-width={16pt},
}

\sisetup{per-mode=symbol}
\footer{}{\thepage}{}

\begin{document}
% \includepdf[pages={1}]{Caratula_Examen_Parcial_03_PU_Fisica_4_01.pdf}

% \newpage
\begin{center}
\textbf{Cada ejercicio vale 1 punto, incluidos los de ejecución.}
\end{center}

\begin{questions}

    \section{Ondas y sonido.}
    
    \vbox{\leftskip\leftmargin A continuación se te presenta la definición de un concepto, la palabra faltante está en una de las opciones de respuesta, relaciona cada definición con la palabra.}

    \question La \rule{2cm}{0.3mm} es el tiempo que se necesita para hacer una oscilación completa.
    \begin{tasks}(4)
        \task Periodo.
        \task Frecuencia.
        \task Elongación.
        \task Amplitud.
    \end{tasks}
    \question Cuando se habla de la distancia que separa dos puntos equivalentes consecutivos de la onda, estamos haciendo referencia a \rule{2cm}{0.3mm} 
    \begin{tasks}(4)
        \task Amplitud.
        \task Longitud de onda.
        \task Cresta.
        \task Periodo.
    \end{tasks}
    \question \rule{2cm}{0.3mm} es el número de oscilaciones o vibraciones que realiza la onda por segundo.
    \begin{tasks}(4)
        \task Amplitud.
        \task Longitud de onda.
        \task Frecuencia.
        \task Periodo.
    \end{tasks}
    \question Cuando se habla de que es el recorrido de la onda desde un punto hasta el siguiente punto equivalente, se está hablando de: \rule{1.5cm}{0.3mm}
    \begin{tasks}(4)
        \task Elongación.
        \task Cresta.
        \task Ciclo.
        \task Valle.
    \end{tasks}
    \question Se definen las ondas longitudinales a aquellas que se generan cuando las partículas del medio material vibra de manera \rule{2cm}{0.3mm} a la dirección de propagación de la onda.
    \begin{tasks}(4)
        \task Uniforme
        \task Perpendicular
        \task Reflejante
        \task Paralela
    \end{tasks}
    \question \textbf{Ejercicio de ejecución.} En una cuerda tensa se producen ondas con una frecuencia de \SI{240}{\hertz}, a una velocidad de propagación cuya magnitud es de \SI{150}{\meter\per\second}. ¿Qué longitud de onda tienen?
    \begin{tasks}(4)
       \task \SI{0.625}{\meter}
       \task \SI{0.855}{\meter}
       \task \SI{1.920}{\meter}
       \task \SI{1.102}{\meter}
   \end{tasks}
    \question \textbf{Ejercicio de ejecución.} Determinar cuál es el periodo de las ondas producidas en una cuerda de violín si la velocidad de propagación tiene una magnitud de \SI{220}{\meter\per\second} y su longitud de onda es de $\lambda = \SI{0.2}{\meter}$.
    \begin{tasks}(4)
       \task \SI{9.09d-2}{\second}
       \task \SI{9.09d-3}{\second}
       \task \SI{9.09d-4}{\second}
       \task \SI{9.09d-5}{\second}
    \end{tasks}
%     \question \textbf{Ejercicio de ejecución.} Una fuente sonora produce un sonido con una frecuencia de \SI{750}{\hertz}, calcula su longitud de onda en el agua, considera que la magnitud de la velocidad del sonido en el agua es de \SI{1435}{\meter\per\second}.
%     \begin{tasks}(4)
%        \task \SI{1.813}{\meter}
%        \task \SI{1.913}{\meter}
%        \task \SI{2.856}{\meter}
%        \task \SI{3.102}{\meter}
%    \end{tasks}
   \question \textbf{Ejercicio de ejecución.} ¿Cuál es la intensidad de un sonido de \SI{40}{\dB}?
   \begin{tasks}(4)
       \task \SI{1d-8}{\watt\per\square\meter}
       \task \SI{1d-9}{\watt\per\square\meter}
       \task \SI{1d-10}{\watt\per\square\meter}
       \task \SI{1d-11}{\watt\per\square\meter}
   \end{tasks}
   \question \textbf{Ejercicio de ejecución.} La conversación ordinaria corresponde a un nivel de sonido de aproximadamente \SI{65}{\dB}. Si dos personas hablan al mismo tiempo, el nivel de sonido es:
   \begin{tasks}(4)
       \task \SI{65}{\dB}
       \task \SI{68}{\dB}
       \task \SI{75}{\dB}
       \task \SI{130}{\dB}
   \end{tasks}
   \question \textbf{Ejercicio de ejecución: } Una ambulancia lleva una velocidad cuya magnitud es de \SI{70}{\kilo\meter\per\hour} y su sirena suena con una frecuencia de \SI{830}{\hertz}. ¿Qué frecuencia aparente escucha un observador que está parado, cuando la ambulancia se aleja de él?
   \begin{tasks}(4)
    \task \SI{785.11}{\hertz}
    \task \SI{880.33}{\hertz}
    \task \SI{1000.00}{\hertz}
    \task \SI{1659.19}{\hertz}
\end{tasks}

    \section{Óptica.}

    \question ¿Cuáles son las características de la luz que se cumplen tanto en la teoría corpuscular de la luz y la teoría ondulatoria?
    \begin{parts}
        \part Refracción, Reverberación, Reflexión.
        \part Propagación rectilínea, Reflexión, Refracción.
        \part Reflexión, Resonancia, Propagación rectilínea.
        \part Refracción, Efecto Doppler, Reflexión.
    \end{parts}
    \question De la ley de Snell se nos pide obtener el ángulo $b$ (refractado), ¿Cuál es la expresión que nos resuelve el problema?
    \begin{multicols}{2}
    \begin{tasks}
        \task $ b = \sen^{-1} \left(\dfrac{n_{b} \, \text{sen} \, a}{n_{a}} \right)$
        \task $ b = \sen^{-1} \left(\dfrac{n_{a} \, \text{sen} \, a}{n_{b}} \right)$
        \task $ b = \sen^{-1} \left(\dfrac{n_{a} \, n_{b}}{\text{sen} \, a} \right)$
        \task $ b = \sen^{-1} \left(\dfrac{\text{sen} \, a}{n_{a} \, n_{b}} \right)$
    \end{tasks}
    \end{multicols}
    \question \textbf{Ejercicio de ejecución: } El haz de una linterna incide sobre la superficie de un panel de vidrio $(n = 1.56)$ en un ángulo de \ang{75} con la normal. ¿Cuál es el ángulo de refracción?
    \begin{tasks}(4)
        \task \ang{39;49;51}
        \task \ang{38;15;23}
        \task \ang{48;4;36}
        \task \ang{27;11;58}
    \end{tasks}
    \question Las lentes \rule{2.5cm}{0.1mm} son aquellas cuyo espesor disminuye de los bordes hacia el centro, hablamos de las lentes:
    \begin{tasks}(4)
        \task Esféricas.
        \task Divergentes.
        \task Convergentes.
        \task Cilíndricas.
    \end{tasks}
    \question Revisa con cuidado la siguiente imagen e identifica las siguientes lentes en el orden que se presenta, ya sean positivas o negativas.
    \begin{figure}[H]
        \centering
        \includegraphics[scale=2]{Arreglo_Lentes_02.png}
    \end{figure}
    \begin{tasks}(4)
        \task +, - , +, - 
        \task -, - , +, + 
        \task +, + , -, - 
        \task -, + , -, + 
    \end{tasks}
    \question De la convención de signos para las lentes delgadas:  La distancia de la imagen $d_{i}$ es \rule{2cm}{0.1mm} para una imagen virtual.
    \begin{tasks}(4)
        \task Negativa.
        \task Positiva.
        \task Infinita.
        \task Cero.
    \end{tasks}
    \question \textbf{Ejercicio de ejecución. } Un objeto de \SI{15}{\centi\meter} se coloca a \SI{50}{\centi\meter} de una lente positiva que tiene una distancia focal de \SI{20}{\centi\meter}. ¿De qué tamaño es la imagen?
    \begin{tasks}(4)
        \task \SI{35}{\centi\meter}
        \task \SI{10}{\centi\meter}
        \task \SI{25}{\centi\meter}
        \task \SI{5}{\centi\meter}
    \end{tasks}
    \question \textbf{Ejercicio de ejecución: } ¿Cuál es la potencia de una lente con \SI{15}{\centi\meter} de distancia focal?
    \begin{tasks}(4)
        \task \num{-2.98} Dioptrías
        \task \num{2.98} Dioptrías
        \task \num{6.66} Dioptrías
        \task \num{-5.25} Dioptrías
    \end{tasks}
    \question El siguiente esquema representa un ojo humano, ¿cómo se denomina al caso en términos de anomalías/normalidad de la visión?
    \begin{figure}[H]
        \centering
        \includegraphics[scale=0.3]{Defectos_Vision_03.png}
    \end{figure}
    \begin{tasks}(4)
        \task Ojo astigmata.
        \task Ojo emétrope.
        \task Ojo hipermétrope.
        \task Ojo miope.
    \end{tasks}
    \question  La \rule{2cm}{0.1mm} es la capacidad de utilizar ambos ojos de manera coordinada y simultánea para percibir una sola imagen tridimensional del entorno.
    \begin{tasks}(4)
        \task Visión periférica.
        \task Visión 20/20.
        \task Visión monocular.
        \task Visión binocular.
    \end{tasks}
\end{questions}
%     \section{(6 puntos) Hidrodinámica.}

%     \question \textbf{(1 punto)} En la hidrodinámica un fluido se considera ideal cuando:
%     \begin{parts}
%         \part La densidad es constante.
%         \part Las fuerzas son no conservativas.
%         \part Las partículas solo tienes movimiento de traslación.
%         \part La temperatura del fluido varía con la velocidad.
%     \end{parts}
%     \question \textbf{(1 punto)} \label{Problema_01} \textbf{Ejercicio de ejecución: } En el cuerpo humano el flujo sanguíneo es de $5$ litros de sangre por minuto. ¿Cuál es el área de la sección transversal de la aorta, si la sangre en ese vaso tiene una velocidad de \SI{28}{\centi\meter\per\second}?
%     \begin{tasks}(4)
%         \task \SI{3.10}{\square\centi\meter}
%         \task \SI{2.77}{\square\centi\meter}
%         \task \SI{1.77}{\square\centi\meter}
%         \task \SI{2.97}{\square\centi\meter}
%     \end{tasks}
%     \question \textbf{(1 punto)} Bernoulli al estudiar la relación entre la presión y la velocidad de un líquido que circula por una tubería, encontró que la presión es \rule{2cm}{0.1mm} si su velocidad es \rule{2cm}{0.1mm}:
%     \begin{tasks}(4)
%         \task alta - cero.
%         \task baja - máxima.
%         \task baja - alta.
%         \task alta - alta.
%     \end{tasks}
%     \question \textbf{(1 punto)} La ecuación de Bernoulli que se muestra a continuación contiene tres términos de cada lado de la igualdad:
%     \begin{align*}
%     \underbrace{P_{1}}_{\text{\large{a}}} + \underbrace{\rho \, g \, h_{1}}_{\text{\large{b}}} + \underbrace{\dfrac{1}{2} \rho \, v_{1}^{2}}_{\text{\large{c}}} = P_{2} + \rho \, g \, h_{2} + \dfrac{1}{2} \rho \, v_{2}^{2}
%     \end{align*}
        
%     \vspace{0.5em}
%     \begin{inparaenum}[I)]
%             \item Energía cinética. \quad \quad
%             \item Energía potencial. \quad \quad
%             \item Potencia. \quad \quad
%             \item Presión.
%     \end{inparaenum}

%     Selecciona la respuesta que relaciona los tres términos con las cantidades físicas correspondientes.
%     \begin{tasks}
%         \task a - III, b - I, c - II
%         \task a - IV, b - I, c - II
%         \task a - IV, b - II, c - I
%         \task a - III, b - II, c - I
%     \end{tasks}
%     \question \textbf{(1 punto)} En la siguiente figura se presenta el flujo \rule{2cm}{0.1mm} en un conducto.
%     \begin{figure}[H]
%         \centering
%         \includegraphics[scale=0.8]{Flujo_02_Laminar.png}
%     \end{figure}
%     \begin{tasks}(4)
%         \task laminar
%         \task turbulento
%         \task seccionado
%         \task estático
%     \end{tasks}
%     \question \textbf{(1 punto)} El número de Reynolds es una cantidad adimensional que nos indica cuándo un flujo \rule{2cm}{0.1mm} pasa a ser un flujo \rule{2cm}{0.1mm}.
%     \begin{tasks}
%         \task laminar - seccionado
%         \task estático - turbulento
%         \task laminar  - turbulento
%         \task turbulento - laminar
%     \end{tasks}
    
%     \section{(9 puntos) Electricidad.}

%     \question \textbf{(1 punto)} El flujo de las partículas cargadas en un conductor es lo que se conoce como \rule{2cm}{0.1mm}.
%     \begin{tasks}(4)
%         \task Resistencia.
%         \task Voltaje.
%         \task Potencia.
%         \task Corriente.
%     \end{tasks}
%     \question \textbf{(1 punto)} Relaciona las unidades de las siguientes cantidades, una columna con la otra.
%     \\
%     \begin{minipage}[t]{0.4\linewidth}
%         \begin{enumerate}[label=\arabic*)]
%             \item Voltaje.
%             \item Resistencia.
%             \item Corriente.
%         \end{enumerate}
%     \end{minipage}
%     \begin{minipage}[t]{0.4\linewidth}
%         \begin{enumerate}[label=\alph*)]
%             \item Ohm (\si{\ohm})
%             \item Watt (\si{\watt})
%             \item Ampere (\si{\ampere})
%             \item Volt (\si{\volt})
%         \end{enumerate}
%     \end{minipage}
%     \begin{tasks}(4)
%         \task 1-d, 2-b, 3-a
%         \task 1-a, 2-d, 3-c
%         \task 1-d, 2-a, 3-c
%         \task 1-a, 2-c, 3-b
%     \end{tasks}
%     \question \textbf{(1 punto)} \label{Problema_02} \textbf{Ejercicio de ejecución: } Determina la intensidad de la corriente eléctrica a través de una resistencia de \SI{150}{\kilo\ohm} al aplicarle una diferencia de potencial de \SI{220}{\volt}.
%     \begin{tasks}(4)
%         \task \SI{1.466}{\milli\ampere}
%         \task \SI{1.966}{\ampere}
%         \task \SI{2.066}{\milli\ampere}
%         \task \SI{2.066}{\ampere}
%     \end{tasks}
%     \question \textbf{(1 punto)} En el siguiente circuito eléctrico se tienen $n$ resistencias del mismo valor conectadas en serie, ¿cuál es el valor de la resistencia total $(R_{T})$ del circuito entre los puntos $A^{\prime}$ y $B^{\prime}$?
%     \begin{center}
%     \begin{circuitikz}[american voltages]
%         \draw 
%             (0, 0) node [anchor=east] {$A^{\prime}$}
%             to[short, o-] (1, 0)
%             % (0, 0) to[V=10V] (0, 4)
%             to [R, l=\mbox{$R$}] (3, 0)
%             to [R, l=\mbox{$R$}] (5, 0)
%             to [R, l=\mbox{$R$}] (7, 0)
%             -- (8.5, 0) node[midway,fill=white,inner sep=5,scale=1.2] {$.\,.\,.$} (10.5, 0)
%             ;
%             \draw (8.5, 0) to [R, l=\mbox{$R$}] (10.5, 0)
%                 to[short, -o] (12, 0)
%                 node [anchor=west] {$B^{\prime}$};
%     \end{circuitikz}  
%     \end{center}
%     \begin{tasks}(4)
%         \task $R_{T} = n^{2} \, R$
%         \task $R_{T} = n + R$
%         \task $R_{T} = n \, R$
%         \task $R_{T} = \dfrac{R}{n}$
%     \end{tasks}
%     \question \textbf{(1 punto)} En un circuito serie $RC$, se tiene que $R = \SI{5}{\kilo\ohm}$ y $C = \SI{3}{\micro\farad}$. ¿Cuál es la constante de tiempo del circuito?
%     \begin{tasks}(4)
%         \task \SI{15}{\second}
%         \task \SI{1.5d-3}{\second}
%         \task \SI{0.015}{\second}
%         \task \SI{0.35}{\second}
%     \end{tasks}
%     \question \textbf{(1 punto)} En un circuito \rule{2cm}{0.1mm} la corriente se adelanta \ang{90} con respecto al voltaje:
%     \begin{tasks}(4)
%         \task $R \, C$
%         \task $R \, L$
%         \task $R \, L \, C$
%         \task $R$
%     \end{tasks}
%     \question \textbf{(1 punto)} \label{Problema_03} \textbf{Ejercicio de ejecución: } Calcula la reactancia inductiva del siguiente inductor:
%     \begin{center}
%         \begin{circuitikz}[american voltages]
%             \draw 
%                 (0, 0) to[short, o-] (1, 0)
%                 to [L, l=\mbox{$L=\SI{1.2}{\henry}$}] (3, 0)
%                 to[short, -o] (4, 0);
%             \node at (2, -0.75) {$\omega = \SI{377}{\radian\per\second}$};
%         \end{circuitikz}  
%     \end{center}
%     \begin{tasks}(4)
%         \task \SI{500.9}{\ohm}
%         \task \SI{452.4}{\ohm}
%         \task \SI{370.7}{\ohm}
%         \task \SI{246.1}{\ohm}
%     \end{tasks}
%     \question \textbf{(1 punto)} En un circuito en serie una resistencia $R = \SI{8}{\ohm}$ y un inductor $L = \SI{0.02}{\henry}$ están conectados a una fuente de voltaje $v = \num 283 \, \sin 300 \, t$ Volts. ¿Cuál es el valor de la impedancia compleja?
%     \begin{tasks}(4)
%         \task $Z {=} \SI{16}{\ohm} + j \, \SI{16}{\ohm}$
%         \task $Z {=} \SI{16}{\ohm} - j \, \SI{16}{\ohm}$
%         \task $Z {=} \SI{8}{\ohm} - j \, \SI{6}{\ohm}$
%         \task $Z {=} \SI{8}{\ohm} + j \, \SI{6}{\ohm}$
%     \end{tasks}
%     \question \textbf{(1 punto)} Una resistencia $R = \SI{27.5}{\ohm}$ y un capacitor $C = \SI{66.7}{\micro\farad}$ se conectan en serie. La diferencia de potencial es $V = \num{50} \, \cos 1500 \, t$ \, Volts. ¿Cuál es la magnitud de la impedancia $Z$?
%     \begin{tasks}(4)
%         \task \SI{14.99}{\ohm}
%         \task \SI{980.95}{\ohm}
%         \task \SI{32.32}{\ohm}
%         \task \SI{224.70}{\ohm}
%     \end{tasks}

% \end{questions}

% \vspace*{1cm}
% \textbf{\huge{Formulario.}}
% \begin{table}[H]
%     \centering
%     \setlength{\tabcolsep}{40pt}
%     \renewcommand{\arraystretch}{2.5}
%     \begin{tabular}{c  c}
%         \multicolumn{2}{c}{Hidrodinámica} \\
%         $G = \dfrac{V}{t}$ & $G = A \, v$ \\
%         $\text{Flujo} = \dfrac{m}{t}$ & $\text{Flujo} = \dfrac{\rho \, V}{t}$ \\
%         $\text{Flujo} = G \, \rho$ & \\
%         $v_{1} \, A_{1} = v_{2} \, A_{2}$ & $v = \sqrt{2 \, g \, h}$ \\
%         \multicolumn{2}{c}{$P_{1} + \rho \, g \, h_{1} + \dfrac{1}{2} \rho \, v_{1}^{2} = P_{2} + \rho \, g \, h_{2} + \dfrac{1}{2} \rho \, v_{2}^{2}$} \\ \hline
%         \multicolumn{2}{c}{Electricidad} \\
%         $V = I \, R$ & $R_{T} = R_{1} + R_{2} + R_{3} + \ldots$ \\
%         $\dfrac{1}{R_{T}} = \dfrac{1}{R_{1}} + \dfrac{1}{R_{2}} + \dfrac{1}{R_{3}} + \ldots$ & $\tau = R \, C$ \\
%         $X_{C} = \dfrac{1}{\omega \, C}$ & $X_{L} = \omega\, L$ \\
%         $Z = R + j \, X_{L}$ & $Z = R - j \, X_{C}$ \\
%         $\abs{Z} = \sqrt{R^{2} + (X_{L})^{2}}$ & $\abs{Z} = \sqrt{R^{2} + (X_{C})^{2}}$ \\
% \end{tabular}
% \end{table}

% \newpage

% En este espacio deberás de incluir el desarrollo completo de los Problemas de Ejecución. El problema se califica de la siguiente manera: \textbf{a) Datos: 0.25 puntos}, \textbf{b) Expresión(es): 0.25 puntos}, \textbf{c) Sustitución: 0.25 puntos} y \textbf{d) Manejo de unidades: 0.25 puntos}.

% \vspace*{0.5cm}
% Solución al Problema de Ejecución \ref{Problema_01}:

% \vspace*{4cm}
% \rule{0.9\textwidth}{0.3mm}

% Solución al Problema de Ejecución \ref{Problema_02}:

% \vspace*{4.5cm}
% \rule{0.9\textwidth}{0.3mm}

% Solución al Problema de Ejecución \ref{Problema_03}:


\end{document}