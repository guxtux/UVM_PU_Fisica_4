\documentclass[14pt]{extarticle}
\usepackage[utf8]{inputenc}
\usepackage[T1]{fontenc}
\usepackage[spanish,es-lcroman]{babel}
\usepackage{amsmath}
\usepackage{amsthm}
\usepackage{physics}
\usepackage{tikz}
\usepackage{float}
\usepackage[autostyle,spanish=mexican]{csquotes}
\usepackage[per-mode=symbol]{siunitx}
\usepackage{gensymb}
\usepackage{multicol}
\usepackage{enumitem}
\usepackage[left=2.00cm, right=2.00cm, top=2.00cm, 
     bottom=2.00cm]{geometry}
\usepackage{Estilos/ColoresLatex}
\usepackage{makecell}

\newcommand{\textocolor}[2]{\textbf{\textcolor{#1}{#2}}}
\sisetup{per-mode=symbol}
\DeclareSIUnit[number-unit-product = {\,}]\cal{cal}
\DeclareSIUnit{\dB}{dB}
%\renewcommand{\questionlabel}{\thequestion)}
\decimalpoint
\sisetup{bracket-numbers = false}

\title{\vspace*{-2cm} Actividad 3 - Refracción de la luz y formación de imágenes \vspace{-5ex}}
\date{}

\begin{document}
\maketitle

Esta actividad otorgará hasta \textbf{5 puntos} de Evaluación Continua.
\vspace*{0.5cm}

\textbf{Instrucciones: }
\begin{itemize}
\item Anota tu nombre en cada hoja que ocupes para resolver los ejercicios.
\item Identifica el ejercicio que resuelves.
\item Resuelve detalladamente cada ejercicio, en caso de que no se tenga el desarrollo, no se tomará en cuenta como ejercicio completo, revisa con cuidado el manejo de las unidades.
\item En caso de plagios, se cancelarán todos los trabajos involucrados.
\end{itemize}


\section*{Ejercicios a resolver.}

% Giancolli Vol. 2, 32-5 Ley de Snell
\begin{enumerate}
\item Un buzo hace brillar una linterna hacia arriba desde abajo del agua en un ángulo de \ang{38.5} con la vertical. ¿En qué ángulo sale la luz del agua?
\item El haz de una linterna incide sobre la superficie de un panel de vidrio $(n = 1.56)$ en un ángulo de \ang{63} con la normal. ¿Cuál es el ángulo de refracción?
\item Se observa que los rayos del Sol forman un ángulo de \ang{33.0} con la vertical debajo del agua. ¿En qué ángulo sobre el horizonte está el Sol?
% Pérez. Unidad 16
\item Un objeto de \SI{4}{\centi\meter} se coloca a una distancia de \SI{13}{\centi\meter} de una lente convergente cuya distancia focal es de \SI{8}{\centi\meter}. Calcula:
\begin{enumerate}
\item ¿A qué distancia de la lente se forma la imagen?
\item ¿Cuál es su tamaño?
\end{enumerate}
\item Un objeto de \SI{5}{\centi\meter} se coloca a \SI{6}{\centi\meter} de una lente divergente que tiene una distancia focal de \SI{9}{\centi\meter}. Calcula:
\begin{enumerate}
\item ¿A qué distancia se forma la imagen de la lente?
\item ¿Qué tamaño tiene?
\end{enumerate}
\end{enumerate}
\end{document}