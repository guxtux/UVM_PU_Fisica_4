\documentclass[14pt]{extarticle}
\usepackage[utf8]{inputenc}
\usepackage[T1]{fontenc}
\usepackage[spanish,es-lcroman]{babel}
\usepackage{amsmath}
\usepackage{amsthm}
\usepackage{physics}
\usepackage{tikz}
\usepackage{float}
\usepackage[autostyle,spanish=mexican]{csquotes}
\usepackage[per-mode=symbol]{siunitx}
\usepackage{gensymb}
\usepackage{multicol}
\usepackage{enumitem}
\usepackage{circuitikz}
\usepackage[left=2.00cm, right=2.00cm, top=2.00cm, 
     bottom=2.00cm]{geometry}
\usepackage{makecell}

\newcommand{\textocolor}[2]{\textbf{\textcolor{#1}{#2}}}
\sisetup{per-mode=symbol}
\DeclareSIUnit[number-unit-product = {\,}]\cal{cal}
\DeclareSIUnit{\dB}{dB}
%\renewcommand{\questionlabel}{\thequestion)}
\decimalpoint
\sisetup{bracket-numbers = false}

\title{\vspace*{-2cm} Actividad Grupo 93 \\  Evaluación Continua - Física IV\vspace{-5ex}}
\date{16 de febrero de 2024}

\begin{document}
\maketitle

\textbf{Nombre:} \rule{8cm}{0.1mm}

\vspace*{0.75cm}
\textbf{Importante: } La hoja deberás de devolverla antes de que concluya la clase, ya que será la evidencia de tu asistencia a la clase de Física IV.

\vspace*{0.5cm}
Esta actividad otorgará hasta \textbf{4 puntos}. Deberás de entregar el desarrollo completo mostrando los pasos para obtener las resistencias equivalentes (si necesitas más hojas, agrégalas a esta misma, anotando tu nombre), si se reporta el resultado directo, sin presentar el desarrollo, sin manejo de unidades en cada paso, no se contará como ejercicio resuelto.

\vspace*{0.75cm}
Podrás apoyarte con tus notas.

\section{Ejercicios.}

\begin{enumerate}
\item Resuelve el siguiente circuito mixto con resistencias. Calcula el valor de la resistencia total y la corriente total. El valor de todas las resistencias es de \SI{25}{\ohm}.
\begin{figure}[H]
\centering
\begin{circuitikz}
    \draw (0, -1.5) to[battery1={50 V}] (0, 6)
        to[R=$R_{1}$] (3, 6) coordinate (a)
        to[R=$R_{4}$] (8, 6)
        to[short] (8, 5)
        to[short] (7, 5) |- (9, 5)
        to[R=$R_{6}$] (9, 2.5)
        to[short] (8, 2.5)
        to[short] (8, 1.5)
        to[short] (7, 1.5) |- (9, 1.5)
        to[R=$R_{8}$] (9, -0.5)
        to[short] (8, -0.5)
        to[short] (8, -1.5)
        to[short] (3, -1.5) coordinate (b)
        to[R=$R_{3}$] (0, -1.5);

    \draw (a) to[R, l_=$R_{2}$] (b);
    
    \draw (7, 5) to[R, l_=$R_{5}$] (7, 2.5)
        to[short] (8, 2.5);
    
    \draw (7, 1.5) to[R, l_=$R_{7}$] (7, -0.5)
        to[short] (8, -0.5);

\end{circuitikz}
\end{figure}
\item Completa la tabla con los datos que hacen falta.
\begin{table}[H]
\centering
\renewcommand{\arraystretch}{1.1}
\begin{tabular}{| c | c  | c |} \hline
Resistencia & Voltaje & Corriente \\ \hline
$R_{1}$ & & \\ \hline
$R_{2}$ & & \\ \hline
$R_{3}$ & & \\ \hline
$R_{4}$ & & \\ \hline
$R_{5}$ & & \\ \hline
$R_{6}$ & & \\ \hline
$R_{7}$ & & \\ \hline
$R_{8}$ & & \\ \hline
\end{tabular}
\end{table}
\end{enumerate}

\end{document}