\documentclass[12pt]{exam}
\usepackage[utf8]{inputenc}
\usepackage[T1]{fontenc}
\usepackage[spanish]{babel}
\usepackage{amsmath}
\usepackage{amsthm}
\usepackage{physics}
\usepackage{tikz}
\usepackage{float}
\usepackage{siunitx}
\usepackage{multicol}
\usepackage[left=2.00cm, right=2.00cm, top=2.00cm, 
     bottom=2.00cm]{geometry}

\renewcommand{\questionlabel}{\thequestion)}
\decimalpoint

\setlength{\belowdisplayskip}{-0.5pt}
\begin{document}

\textbf{Matemáticas.}
\begin{multicols}{2}
\begin{questions}
     \question Si $4$ libros cuestan $\$200$ pesos, ¿Cuánto costarán 3 docenas de libros?
     \begin{choices}
         \choice $\$1800$
         \choice $\$1900$
         \choice $\$2000$
         \choice $\$2100$
     \end{choices}
     %\answerline
     \question $4$ hombres hacen una obra en $12$ días. ¿En cuántos días podrán hacer la obra $7$ hombres?
     \begin{choices}
        \choice $5 \, \dfrac{1}{2}$ días.
        \choice $6 \, \dfrac{7}{6}$ días.
        \choice $6$ días.
        \choice $6 \, \dfrac{6}{7}$ días.
    \end{choices}
    %\answerline
    \question $3$ hombres trabajando $8$ horas diarias han hecho $80$ metros de obra en $10$ días. ¿Cuántos días necesitarán $5$ hombres, trabajando $6$ horas diarias, para hacer $60$ metros de la misma obra?
     \begin{choices}
        \choice $6$ días.
        \choice $11$ días.
        \choice $8$ días.
        \choice $7$ días.
    \end{choices}
    %\answerline
    \question Un campamento con $1600$ personas tiene víveres para $10$ días si consumen $3$ raciones diarias por persona. Reciben a $400$ personas más, ¿cuántos días durarán los víveres si cada persona toma $2$ raciones diarias?
     \begin{choices}
        \choice $10$ días.
        \choice $11$ días.
        \choice $12$ días.
        \choice $13$ días.
    \end{choices}
    %\answerline
    \question Si $4$ libros cuestan $200$ pesos, ¿cuántos costarán $3$ docenas de libros?
     \begin{choices}
        \choice $\$2000$ 
        \choice $\$1950$ 
        \choice $\$1900$ 
        \choice $\$1800$
    \end{choices}
    %\answerline
    \question Una torre de $\SI{25.05}{\meter}$ da una sombra de $\SI{33.40}{\meter}$ ¿Cuál será a la misma hora, la sombra de una persona cuya estatura es de $\SI{1.80}{\meter}$
     \begin{choices}
        \choice $\SI{3.30}{\meter}$ 
        \choice $\SI{2.77}{\meter}$ 
        \choice $\SI{12}{\meter}$ 
        \choice $\SI{2.40}{\meter}$
    \end{choices}
    %\answerline
    \question Una cuadrilla de obreros emplea $14$ días trabajando $8$ horas diarias para realizar una obra. Su hubieran trabajado una hora menos al día, ¿en cuántos días habrían terminado la obra?
     \begin{choices}
        \choice $18$ días.
        \choice $16$ días.
        \choice $20$ días.
        \choice $21$ días.
    \end{choices}
    %\answerline
    \question Con una velocidad de $\SI[per-mode=symbol]{30}{\kilo\meter\per\hour}$ tarda $8 \dfrac{1}{4}$ horas en ir de una ciudad a otra. ¿Cuánto tiempo menos se hubiera tardado si la velocidad hubiera sido el triple?
     \begin{choices}
        \choice $\SI{7.5}{\hour}$ 
        \choice $\SI{6.5}{\hour}$ 
        \choice $\SI{5.5}{\hour}$ 
        \choice $\SI{4.5}{\hour}$
    \end{choices}
    %\answerline
    \question Una pieza de tela tiene $\SI{32.32}{\meter}$ de largo y $\SI{75}{\centi\meter}$ de ancho. ¿Cuál será la longitud, para que sin variar la superficie, el ancho sea de $\SI{2}{\meter}$?
     \begin{choices}
        \choice $\SI{3.25}{\meter}$ 
        \choice $\SI{1.25}{\meter}$ 
        \choice $\SI{1.5}{\meter}$ 
        \choice $\SI{2.5}{\meter}$
    \end{choices}
    %\answerline
    \newpage
    \question Un obrero tarda $12 \dfrac{3}{5}$ días en hacer $\dfrac{7}{12}$ de una obra. ¿Cuánto tiempo necesitará para termina la obra?
     \begin{choices}
        \choice $9 \dfrac{3}{4}$ días. 
        \choice $9 \dfrac{11}{12}$ días. 
        \choice $9 \dfrac{1}{2}$ días. 
        \choice $9$ días.
    \end{choices}
    %\answerline
    \question Resuelve la ecuación:
    \begingroup
    \abovedisplayskip=0pt
    \belowdisplayskip=-10pt
    \begin{align*}
    5 \, x = 8 \, x - 15
    \end{align*}
    \endgroup
    \begin{choices}
        \choice $x = -5$
        \choice $x = -3$
        \choice $x = 3$
        \choice $x = 5$
    \end{choices}
    %\answerline
    \question Resuelve la ecuación:
    \begingroup
    \abovedisplayskip=0pt
    \belowdisplayskip=-10pt
    \begin{align*}
    11 \, x + 5 \, x - 1 = 65 \, x - 3
    \end{align*}
    \endgroup
    \begin{choices}
        \choice $x = \dfrac{3}{7}$
        \choice $x = \dfrac{4}{7}$
        \choice $x = \dfrac{5}{7}$
        \choice $x = \dfrac{6}{7}$
    \end{choices}
    %\answerline
    \question Resuelve la ecuación:
    \begingroup
    \abovedisplayskip=0pt
    \belowdisplayskip=-10pt
    \begin{align*}
    x - ( 2 \, x) = 8 - (3 \, x + 3)
    \end{align*}
    \endgroup
    \begin{choices}
        \choice $x = 0$
        \choice $x = 1$
        \choice $x = 2$
        \choice $x = 3$
    \end{choices}
    %\answerline
    \question La suma de edad de $A$, $B$ y $C$ es $69$ años. La edad de $A$ es doble que la de $B$ y $6$ años mayor que la de $C$. ¿Cuál es la edad en años de $A$, $B$ y $C$.
    \begin{choices}
        \choice $A = 29$, $B = 16$, $C = 23$.
        \choice $A = 31$, $B = 15$, $C = 24$.
        \choice $A = 28$, $B = 18$, $C = 24$.
        \choice $A = 30$, $B = 15$, $C = 24$.
    \end{choices}
    \question ¿Cuál es el mínimo común múltiplo de los siguientes números?
    \begingroup
    \abovedisplayskip=0pt
    \belowdisplayskip=-10pt
    \begin{align*}
    5, \quad 10, \quad 15, \quad 30, \quad 45
    \end{align*}
    \begin{choices}
        \choice $18$
        \choice $50$
        \choice $60$
        \choice $90$
    \end{choices}
\end{questions}

\end{multicols}
\end{document}