\documentclass[14pt]{extarticle}
\usepackage[utf8]{inputenc}
\usepackage[T1]{fontenc}
\usepackage[spanish,es-lcroman]{babel}
\usepackage{amsmath}
\usepackage{amsthm}
\usepackage{physics}
\usepackage{tikz}
\usepackage{float}
\usepackage[autostyle,spanish=mexican]{csquotes}
\usepackage[per-mode=symbol]{siunitx}
\usepackage{gensymb}
\usepackage{multicol}
\usepackage{enumitem}
\usepackage[left=2.00cm, right=2.00cm, top=2.00cm, 
     bottom=2.00cm]{geometry}
\usepackage{Estilos/ColoresLatex}
\usepackage{makecell}

\newcommand{\textocolor}[2]{\textbf{\textcolor{#1}{#2}}}
\DeclareSIUnit[number-unit-product = {\,}]\cal{cal}

%\renewcommand{\questionlabel}{\thequestion)}
\decimalpoint
\sisetup{bracket-numbers = false}

\title{\vspace*{-2cm} Actividad 1 - Propiedades de las ondas\vspace{-5ex}}
\date{}

\begin{document}
\maketitle

Esta actividad otorgará hasta \textbf{8 puntos} de Evaluación Continua.
\vspace*{0.5cm}

\textbf{Instrucciones: }
\begin{itemize}
\item Anota tu nombre en cada hoja que ocupes para resolver los ejercicios.
\item Identifica el ejercicio que resuelves.
\item Resuelve detalladamente cada ejercicio, en caso de que no se tenga el desarrollo, no se tomará en cuenta como ejercicio completo, revisa con cuidado el manejo de las unidades.
\item En caso de plagios, se cancelarán todos los trabajos involucrados.
\end{itemize}

\section*{Ejercicios a resolver.}

\begin{enumerate}
\item Calcular la magnitud de la velocidad con la que se propaga una onda cuya frecuencia es de \SI{150}{\hertz} y su longitud de onda es de \SI{7}{\meter}.
\item Una lancha sube y baja por el paso de las olas cada \SI{4}{\second}, entre cresta y cresta hay una distancia de \SI{15}{\meter}. ¿Cuál es la magnitud de la velocidad con que se mueven las olas?
\item La cresta de una onda producida en la superficie libre de un líquido avanza \SI{0.5}{\meter\per\second}. Si tiene una longitud de onda de \SI{4d-1}{\meter}, calcula su frecuencia.
\item Por una cuerda tensa se propagan ondas con una frecuencia de \SI{30}{\hertz} y una rapidez de propagación de \SI{10}{\meter\per\second}. ¿Cuál es su longitud de onda?
\item Calcula la frecuencia y el periodo de las ondas producidas en una cuerda de guitarra, si tienen una rapidez de propagación de \SI{12}{\meter\per\second} y su longitud de onda es de \SI{0.06}{\meter}.
\item Determina la frecuencia de las ondas que se transmiten por una cuerda tensa, cuya rapidez de propagación es de \SI{200}{\meter\per\second} y su longitud de onda es de \SI{0.7}{\meter}.
\item ¿Cuál es la rapidez con que se propaga una onda longitudinal en un resorte, cuando su frecuencia es de \SI{180}{\hertz} y su longitud de onda es de \SI{0.8}{\meter}?
\item Se produce un tren de ondas en una cuba de ondas, entre cresta y cresta hay una distancia de \SI{0.03}{\meter}, con una frecuencia de \SI{90}{\hertz}. ¿Cuál es la magnitud de la velocidad de propagación de las ondas?
\end{enumerate}
\end{document}