\documentclass[14pt]{extarticle}
\usepackage[utf8]{inputenc}
\usepackage[T1]{fontenc}
\usepackage[spanish,es-lcroman]{babel}
\usepackage{amsmath}
\usepackage{amsthm}
\usepackage{physics}
\usepackage{tikz}
\usepackage{float}
\usepackage[autostyle,spanish=mexican]{csquotes}
\usepackage[per-mode=symbol]{siunitx}
\usepackage{gensymb}
\usepackage{multicol}
\usepackage{enumitem}
\usepackage[left=2.00cm, right=2.00cm, top=2.00cm, 
     bottom=2.00cm]{geometry}

%\renewcommand{\questionlabel}{\thequestion)}
\decimalpoint
\sisetup{bracket-numbers = false}

\title{\vspace*{-2cm} Práctica 6 - Ley de Ohm\\  Física IV (Área II) \vspace{-5ex}}
\date{}

\begin{document}
\maketitle

\noindent
\textbf{Nombre:} \rule{8cm}{0.1mm} \textbf{Fecha:} \rule{3cm}{0.1mm} \\[0.5em]
\textbf{Firma Profesor:} \rule{4cm}{0.1mm}

\section{Resistencias individuales.}

\begin{enumerate}
\item En esta sesión tendrás disponibles tres resistencias eléctricas, mide con el multímetro su valor y anótalo en la siguiente tabla, revisa con cuidado si tienes un prefijo en la pantalla del aparato:
\begin{table}[H]
\centering
\begin{tabular}{| c | c |} \hline
Resistencia & Valor multímetro $(\Omega)$ \\ \hline
$R_{1}$ & \\ \hline
$R_{2}$ & \\ \hline
$R_{3}$ & \\ \hline
\end{tabular}
\end{table}
\item Mide el valor de voltaje de corriente directa de la pila que se te entregó, anota su valor: \rule{2cm}{0.1mm}
\item Selecciona el dial del multímetro en corriente directa. Conecta en serie la pila, una de las resistencias y el multímetro, sigue con cuidado la indicación ya que de lo contrario el multímetro se puede dañar de manera permanente.
\item Anota en la siguiente tabla el valor de corriente para esa primera resistencia, repite el procedimiento de registro de corriente eléctrica para las otras dos resistencias.
\begin{table}[H]
\centering
\begin{tabular}{| c | c | c | c |} \hline
Resistencia & R $(\Omega)$ & I multímetro $(\si{\ampere})$ & I fórmula $(\si{\ampere}) $\\ \hline
$R_{1}$ & & & \\ \hline
$R_{2}$ & & & \\ \hline
$R_{3}$ & & & \\ \hline
\end{tabular}
\end{table}
\item Con la expresión de la ley de Ohm, calcula el valor de la corriente eléctrica para cada resistencia, anota ese valor y completa la tabla anterior.
\end{enumerate}

\section{Tres resistencias en serie.}

Coloca las tres resistencias en el protoboard de tal manera que tengas una conexión en serie. Con el multímetro:
\begin{enumerate}
\item Mide el valor de resistencia eléctrica, anota el valor: \rule{2cm}{0.1mm}
\item Mide el valor de la corriente eléctrica del circuito, anota el valor: \rule{2cm}{0.1mm}
\end{enumerate}
Calcula con la fórmula de la ley de Ohm, el valor de la corriente eléctrica del circuito en serie con las tres resistencias: \rule{2cm}{0.1mm}

\section{Tres resistencias en paralelo.}

Ahora con las tres resistencias elabora un circuito en paralelo. También con el multímetro:
\begin{enumerate}
\item Mide el valor de resistencia eléctrica, anota el valor: \rule{2cm}{0.1mm}
\item Mide el valor de la corriente eléctrica, anota el valor: \rule{2cm}{0.1mm}
\end{enumerate}
Calcula con la fórmula de la ley de Ohm, el valor de la corriente eléctrica del circuito en paralelo con las tres resistencias: \rule{2cm}{0.1mm}

\textbf{Importante:} Los datos recabados corresponden al apartado de Resultados del reporte final de la práctica.

\end{document}