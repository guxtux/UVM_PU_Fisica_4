\documentclass[14pt]{beamer}
\usepackage{./Estilos/BeamerUVM}
\usepackage{./Estilos/ColoresLatex}
\usetheme{Madrid}
\usecolortheme{default}
%\useoutertheme{default}
\setbeamercovered{invisible}
% or whatever (possibly just delete it)
\setbeamertemplate{section in toc}[sections numbered]
\setbeamertemplate{subsection in toc}[subsections numbered]
\setbeamertemplate{subsection in toc}{\leavevmode\leftskip=3.2em\rlap{\hskip-2em\inserttocsectionnumber.\inserttocsubsectionnumber}\inserttocsubsection\par}
% \setbeamercolor{section in toc}{fg=blue}
% \setbeamercolor{subsection in toc}{fg=blue}
% \setbeamercolor{frametitle}{fg=blue}
\setbeamertemplate{caption}[numbered]

\setbeamertemplate{footline}
\beamertemplatenavigationsymbolsempty
\setbeamertemplate{headline}{}


\makeatletter
% \setbeamercolor{section in foot}{bg=gray!30, fg=black!90!orange}
% \setbeamercolor{subsection in foot}{bg=blue!30}
% \setbeamercolor{date in foot}{bg=black}
\setbeamertemplate{footline}
{
  \leavevmode%
  \hbox{%
  \begin{beamercolorbox}[wd=.333333\paperwidth,ht=2.25ex,dp=1ex,center]{section in foot}%
    \usebeamerfont{section in foot} {\insertsection}
  \end{beamercolorbox}%
  \begin{beamercolorbox}[wd=.333333\paperwidth,ht=2.25ex,dp=1ex,center]{subsection in foot}%
    \usebeamerfont{subsection in foot}  \insertsubsection
  \end{beamercolorbox}%
  \begin{beamercolorbox}[wd=.333333\paperwidth,ht=2.25ex,dp=1ex,right]{date in head/foot}%
    \usebeamerfont{date in head/foot} \insertshortdate{} \hspace*{2em}
    \insertframenumber{} / \inserttotalframenumber \hspace*{2ex} 
  \end{beamercolorbox}}%
  \vskip0pt%
}
\makeatother

\makeatletter
\patchcmd{\beamer@sectionintoc}{\vskip1.5em}{\vskip0.8em}{}{}
\makeatother

% \usefonttheme{serif}
\usepackage[clock]{ifsym}

\sisetup{per-mode=symbol}
\resetcounteronoverlays{saveenumi}

\title{\Large{Solución a los ejercicios} \\ \normalsize{Física IV (área II)}}
\date{}

% Macro para agregar el logo de UVM en cada slide de la presentación

\addtobeamertemplate{frametitle}{}{%
\begin{tikzpicture}[remember picture,overlay]
\coordinate (logo) at ([xshift=-1.5cm,yshift=-0.8cm]current page.north east);
% \fill[devryblue] (logo) circle (.9cm);
% \clip (logo) circle (.75cm);
\node at (logo) {\includegraphics[width=2.1cm]{Imagenes/logo_UVM.png}};
\end{tikzpicture}}


\begin{document}
\maketitle

\section*{Contenido}
\frame[allowframebreaks]{\frametitle{Contenido} \tableofcontents[currentsection, hideallsubsections]}

\section{Solución}
\frame{\tableofcontents[currentsection, hideothersubsections]}
\subsection{Ejercicio 1}

\begin{frame}
\frametitle{Enunciado del Ejercicio 1}
Calcular la magnitud de la velocidad con la que se propaga una onda cuya frecuencia es de \SI{150}{\hertz} y su longitud de onda es de \SI{7}{\meter}.
\end{frame}
\begin{frame}
\frametitle{Solución al Ejercicio 1}
\textocolor{red}{Datos:}
\pause
\begin{eqnarray*}
\begin{aligned}
f &= \SI{150}{\hertz} \\[0.5em] \pause
\lambda &= \SI{7}{\meter} \\[0.5em] \pause
v &= \, ?
\end{aligned}
\end{eqnarray*}
\end{frame}
\begin{frame}
\frametitle{Solución al Ejercicio 1}
\textocolor{red}{Expresión}:
\pause
\begin{align*}
v = \lambda \, f
\end{align*}
\end{frame}
\begin{frame}
\frametitle{Solución al Ejercicio 1}
\textocolor{red}{Sustitución}:
\pause
\begin{eqnarray*}
\begin{aligned}
v = (\SI{7}{\meter})(\SI{150}{\hertz}) = \pause \SI{1050}{\meter\per\second}
\end{aligned}
\end{eqnarray*}
\end{frame}

\subsection{Ejercicio 2}

\begin{frame}
\frametitle{Enunciado del ejercicio 2}
Una lancha sube y baja por el paso de las olas cada \SI{4}{\second}, entre cresta y cresta hay una distancia de \SI{15}{\meter}.
\\
\bigskip
\pause
¿Cuál es la magnitud de la velocidad con que se mueven las olas?
\end{frame}
\begin{frame}
\frametitle{Solución al Ejercicio 2}
\textocolor{red}{Datos:}
\pause
\begin{eqnarray*}
\begin{aligned}
T &= \SI{4}{\second} \\[0.5em] \pause
\lambda &= \SI{15}{\meter} \\[0.5em] \pause
v &= \, ?
\end{aligned}
\end{eqnarray*}
\end{frame}
\begin{frame}
\frametitle{Solución al Ejercicio 2}
\textocolor{red}{Expresión(es)}:
\pause
\begin{eqnarray*}
\begin{aligned}
f &= \dfrac{1}{T} \\[0.5em] \pause
v &= \lambda \, f
\end{aligned}
\end{eqnarray*}
\end{frame}
\begin{frame}
\frametitle{Solución al Ejercicio 2}
\textocolor{red}{Sustitución}:
\pause
\begin{eqnarray*}
\begin{aligned}
f &= \dfrac{1}{\SI{4}{\second}} = \pause \SI{0.25}{\hertz} \\[0.5em] \pause
v &= (\SI{15}{\meter})(\SI{0.25}{\hertz}) = \pause \SI{3.75}{\meter\per\second}
\end{aligned}
\end{eqnarray*}
\end{frame}

\subsection{Ejercicio 3}

\begin{frame}
\frametitle{Enunciado del Ejercicio 3}
La cresta de una onda producida en la superficie libre de un líquido avanza \SI{0.5}{\meter\per\second}. Si tiene una longitud de onda de \SI{4d-1}{\meter}, \pause calcula su frecuencia.
\end{frame}
\begin{frame}
\frametitle{Solución al Ejercicio 3}
\textocolor{red}{Datos:}
\pause
\begin{eqnarray*}
\begin{aligned}
v &= \SI{0.5}{\meter\per\second} \\[0.5em] \pause
\lambda &= \SI{4d-1}{\meter} \\[0.5em] \pause
f &= \, ?
\end{aligned}
\end{eqnarray*}
\end{frame}
\begin{frame}
\frametitle{Solución al Ejercicio 3}
\textocolor{red}{Expresión(es)}:
\pause
\begin{eqnarray*}
\begin{aligned}
v &= \lambda \, f \\[0.5em] \pause
f &= \dfrac{v}{\lambda}
\end{aligned}
\end{eqnarray*}
\end{frame}
\begin{frame}
\frametitle{Solución al Ejercicio 3}
\textocolor{red}{Sustitución}:
\pause
\begin{eqnarray*}
\begin{aligned}
f &= \dfrac{\SI{0.5}{\meter\per\second}}{\SI{4d-1}{\meter}} = \pause \SI{1.25}{\hertz}
\end{aligned}
\end{eqnarray*}
\end{frame}

\subsection{Ejercicio 4}

\begin{frame}
\frametitle{Enunciado del Ejercicio 4}
Por una cuerda tensa se propagan ondas con una frecuencia de \SI{30}{\hertz} y una rapidez de propagación de \SI{10}{\meter\per\second}. 
\\
\bigskip
\pause
¿Cuál es su longitud de onda?
\end{frame}
\begin{frame}
\frametitle{Solución al Ejercicio 4}
\textocolor{red}{Datos:}
\pause
\begin{eqnarray*}
\begin{aligned}
f &= \SI{30}{\hertz} \\[0.5em] \pause
v &= \SI{10}{\meter\per\second} \\[0.5em] \pause
\lambda &= \, ?
\end{aligned}
\end{eqnarray*}
\end{frame}
\begin{frame}
\frametitle{Solución al Ejercicio 4}
\textocolor{red}{Expresión(es)}:
\pause
\begin{eqnarray*}
\begin{aligned}
v &= \lambda \, f \\[0.5em] \pause
\lambda &= \dfrac{v}{f}
\end{aligned}
\end{eqnarray*}
\end{frame}
\begin{frame}
\frametitle{Solución al Ejercicio 4}
\textocolor{red}{Sustitución}:
\pause
\begin{eqnarray*}
\begin{aligned}
\lambda = \dfrac{\SI{10}{\meter\per\second}}{\SI{30}{\hertz}} = \pause \SI{0.33}{\meter}
\end{aligned}
\end{eqnarray*}
\end{frame}

\subsection{Ejercicio 5}

\begin{frame}
\frametitle{Enunciado del Ejercicio 5}
Calcula la frecuencia y el periodo de las ondas producidas en una cuerda de guitarra, si tienen una rapidez de propagación de \SI{12}{\meter\per\second} y su longitud de onda es de \SI{0.06}{\meter}.
\end{frame}

\begin{frame}
\frametitle{Solución al Ejercicio 5}
\textocolor{red}{Datos:}
\pause
\begin{eqnarray*}
\begin{aligned}
v &= \SI{12}{\meter\per\second} \\[0.5em] \pause
\lambda &= \SI{0.06}{\meter} \\[0.5em] \pause
T &= \, ?
\end{aligned}
\end{eqnarray*}
\end{frame}
\begin{frame}
\frametitle{Solución al Ejercicio 5}
\textocolor{red}{Expresión(es)}:
\pause
\begin{eqnarray*}
\begin{aligned}
v &= \lambda \, f \pause \hspace{0.3cm} \Rightarrow \hspace{0.3cm} f = \dfrac{v}{\lambda} \\[0.5em] \pause
T &= \dfrac{1}{f}
\end{aligned}
\end{eqnarray*}
\end{frame}
\begin{frame}
\frametitle{Solución al Ejercicio 5}
\textocolor{red}{Sustitución}:
\pause
\begin{eqnarray*}
\begin{aligned}
f &= \dfrac{\SI{12}{\meter\per\second}}{\SI{0.06}{\meter}} = \pause \SI{200}{\hertz} \\[0.5em] \pause
T &= \dfrac{1}{\SI{200}{\hertz}} = \pause \SI{0.005}{\second}
\end{aligned}
\end{eqnarray*}
\end{frame}

\subsection{Ejercicio 6}

\begin{frame}
\frametitle{Enunciado del Ejercicio 6}
Determina la frecuencia de las ondas que se transmiten por una cuerda tensa, cuya rapidez de propagación es de \SI{200}{\meter\per\second} y su longitud de onda es de \SI{0.7}{\meter}.
\end{frame}

\subsection{Ejercicio 7}

\begin{frame}
\frametitle{Enunciado del Ejercicio 7}
¿Cuál es la rapidez con que se propaga una onda longitudinal en un resorte, cuando su frecuencia es de \SI{180}{\hertz} y su longitud de onda es de \SI{0.8}{\meter}?
\end{frame}

\subsection{Ejercicio 8}

\begin{frame}
\frametitle{Enunciado del Ejercicio 8}
Se produce un tren de ondas en una cuba de ondas, entre cresta y cresta hay una distancia de \SI{0.03}{\meter}, con una frecuencia de \SI{90}{\hertz}.
\\
\bigskip
\pause
¿Cuál es la magnitud de la velocidad de propagación de las ondas?
\end{frame}


\end{document}